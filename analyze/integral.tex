\documentclass[a4paper, 12pt]{article}

\usepackage{cmap}
\usepackage[T2A]{fontenc}
\usepackage[english, russian]{babel}
\usepackage[utf8]{inputenc}
\usepackage[left=2cm,right=1.5cm,top=2cm,bottom=2cm]{geometry}
% \usepackage{mathtext}
\usepackage{amsmath}
\usepackage{amssymb}
\usepackage{etoolbox}
\usepackage{amsthm}
% \usepackage{nicematrix}
% \usepackage{graphicx}
% \usepackage{tikz}
% \usepackage{parskip}
% \graphicspath{/Images/}

%Реализация aug, overbrace и underbrace без nice matrix
\newcommand\aug{\fboxsep=-\fboxrule\!\!\!\fbox{\strut}\!\!\!}
\newcommand\undermat[2]{\makebox[0pt][l]{$\smash{\underbrace
{\phantom{\begin{matrix}#2\end{matrix}}}_{\text{$#1$}}}$}#2}
\newcommand\overmat[2]{\makebox[0pt][l]{$\smash{\overbrace
{\phantom{\begin{matrix}#2\end{matrix}}}^{\text{$#1$}}}$}#2}
\newcommand\tab[1][.5cm]{\hspace*{#1}}
\newcommand\Underset[2]{\underset{\textstyle #1}{#2}}
\newcommand\Overset[2]{\overset{\textstyle #1}{#2}}


\theoremstyle{definition}
\newtheorem*{definition}{Определение}
\newtheorem*{theorem}{Теорема}
\newtheorem*{consequense}{Следствие}
\newtheorem*{lemma}{Лемма}
\newtheorem*{subtheorem}{Утверждение}
\newtheorem*{remark}{Замечание}


\begin{document}
    \fontsize{14pt}{20pt}\selectfont
    \begin{center}
        \begin{Large}            
            \textbf{Таблица интегралов}
        \end{Large}
        \begin{tabular}{l l}\\
            $\int x^adx = \frac{x^{a+1}}{a + 1} + C\ (a \neq -1)$ &
            $\int \frac{dx}{a^2 + x^2} = \frac{1}{a}arctg \frac{x}{a} + C\ (a>0)$\\\\
            $\int \frac{dx}{x} = \ln |x| + 1,\ x \neq 0$ &
            $\int \frac{dx}{a^2 - x^2} = \frac{1}{2a}\ln \frac{|a + 1|}{|a - x|} + C\ (a > 0)$\\\\
            $\int a^xdx = \frac{a^x}{\ln a} + C\ (a > 0, a \neq 1)$ &
            $\int \frac{dx}{\sqrt{a^2 - x^2}} = \arcsin \frac{x}{a} + C\ (a > 0)$\\\\
            $\int e^xdx = e^x + C$ &
            $\int \frac{dx}{\sqrt{x^2 + k}} = \ln|x + \sqrt{x^2 + k}| + C\ (k \neq 0)$\\\\
            $\int \sin xdx = -\cos x + C$ & $\int\sh xdx = \ch x + C$\\\\
            $\int \cos xdx = \sin x + C$ & $\int \ch xdx = \sh x + C$\\\\
            $\int \frac{dx}{\cos^2 x} = \tg x + C$ &
            $\int \frac{dx}{sh^2 x} = -\cth x + C$\\\\
            $\int \frac{dx}{\sin^2 x} = -\ctg x + C$ &
            $\int \frac{dx}{ch^2 x} = \th x + C$
        \end{tabular}     
    \end{center}
    \begin{center}
        \textbf{Методы решения}
    \end{center}
    \underline{По частям}:$\int udv = uv - \int vdu$\\
    $\\$\underline{Метод Остроградского}: $\int \frac{P_n(x)}{Q_m(x)}dx =
    \frac{P_{m-k-1}}{Q_{m-k}} + \int \frac{R_{k-1}(x)}{Q_k(x)}dx$\\
    $\\$\underline{Иррациональные функции}: $\int \frac{P_n(x)}{y}dx =
    Q_{n-1}(x)y + \lambda\int \frac{dx}{y},\ y = \sqrt{ax^2 + by + c}$\\
    $\\$\underline{Освновная тригонометрическая замена:}
    $t = \tg \frac{x}{2},\ \sin x = \frac{2\tg \frac{x}{2}}
    {1 + \tg^2\frac{x}{2} },\ \cos x = \frac{1 - \tg^2 \frac{x}{2}}
    {1 + \tg^2 \frac{x}{2}}$\\$\\$
    \underline{Другие способы:} $\int \frac{a_1\sin x + 
    b_1\cos x + c_1}{a\sin x + b\cos y + c}dx = Ax + B\ln|
    a\sin x + b\cos y + c| +$
    \begin{flushright}
        $+ C\int \frac{dx}{a\sin x + b\cos y + c} $     
    \end{flushright}
    $\int\sin^nxdx = -\frac{1}{n}\cos x\sin^{n-1} x + 
    \frac{n-1}{n}\int\sin^{n-2}xdx$\\
    $\int\cos^nxdx = \frac{1}{n}\sin x\cos^{n-1} x + 
    \frac{n-1}{n}\int\cos^{n-2}xdx$ 

\end{document}