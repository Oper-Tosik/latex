\documentclass[a4paper, 12pt]{article}

\usepackage{cmap}
\usepackage[T2A]{fontenc}
\usepackage[english, russian]{babel}
\usepackage[utf8]{inputenc}
\usepackage[left=2cm,right=1.5cm,top=2cm,bottom=2cm]{geometry}
% \usepackage{mathtext}
\usepackage{amsmath}
\usepackage{amssymb}
\usepackage{etoolbox}
\usepackage{amsthm}
\usepackage{booktabs}
% \usepackage{nicematrix}
\usepackage{graphicx}
% \usepackage{tikz}
% \usepackage{parskip}

%Реализация aug, overbrace и underbrace без nice matrix
\newcommand\aug{\fboxsep=-\fboxrule\!\!\!\fbox{\strut}\!\!\!}
\newcommand\undermat[2]{\makebox[0pt][l]{$\smash{\underbrace
{\phantom{\begin{matrix}#2\end{matrix}}}_{\text{$#1$}}}$}#2}
\newcommand\overmat[2]{\makebox[0pt][l]{$\smash{\overbrace
{\phantom{\begin{matrix}#2\end{matrix}}}^{\text{$#1$}}}$}#2}
\newcommand\tab[1][.5cm]{\hspace*{#1}}
\newcommand\Underset[2]{\underset{\textstyle #1}{#2}}
\newcommand\Overset[2]{\overset{\textstyle #1}{#2}}


\theoremstyle{definition}
\newtheorem*{definition}{Определение}
\newtheorem*{theorem}{Теорема}
\newtheorem*{consequense}{Следствие}
\newtheorem*{lemma}{Лемма}
\newtheorem*{subtheorem}{Утверждение}
\newtheorem*{remark}{Замечание}

\usepackage[russian]{babel}
\addto\captionsenglish{% Replace "english" with the language you use
  \renewcommand{\contentsname}%
    {Содержание}%
}

\usepackage{titlesec}
\titleformat{\section}{\LARGE \bfseries}{\thesection}{1em}{}
\titleformat{\subsection}{\Large\bfseries}{\thesubsection}{1em}{}
\titleformat{\subsubsection}{\large\bfseries}{\thesubsubsection}{1em}{}

\usepackage{hyperref}
\usepackage{xcolor}
% Цвета для гиперссылок
\definecolor{linkcolor}{HTML}{225ae2} % цвет ссылок
\definecolor{urlcolor}{HTML}{225ae2} % цвет гиперссылок
\hypersetup{
    pdfstartview=FitH, 
    linkcolor=linkcolor,
    urlcolor=urlcolor,
    colorlinks=true
}

\title{\textbf{Безопасность жизенедеятельности}}
\author{Ким Никита, 111 группа}

\begin{document}
    \fontsize{14pt}{20pt}\selectfont
    \maketitle
    \newpage
    \tableofcontents
    \fontsize{14pt}{20pt}\selectfont
    \newpage
    \section{Гражданская оборона.}
        \subsection{Сигнал ГО, предназначенный для оповещения населения о ЧС.}
        В РФ с 2020г. установлен \textbf{единый сигнал ГО "ВНИМАНИЕ ВСЕМ!"}, который сопровождается \textbf{включением сирен, прерывистыми гудками с последующей речевой информацией о сложившейся ситуации} (о химическом заражении, радиационной опасности, воздушной тревоге и других опасностях) и \textbf{порядке действий населения.}  
        \begin{definition}
            Оповещение населения о чрезвычайных ситуациях -- это доведение до населения сигнала оповещения и экстренной информации об опасностях, возникающих при угрозе возникновения или возникновения чрезвычайных ситуаций природного или техногенного характера, \textbf{О правилах проведения населения и необходимости проведения мероприятий по защите. (04.11.2022 N 217-ФЗ).} 
        \end{definition}
        \begin{definition}
            Информирование населения о чрезвычайных ситуациях -- это доведение до населения через средства массовой информации и по иным каналам информации о прогнозируемых и возникших ЧС, принимаемых мерах по обеспечению безопасности населения и территорий, способах защиты, а также проведение пропаганды знаний в области ГО, защиты населения и территорий от чрезвычайных ситуаций, в том числе обеспечения безопасности людей на водных объектах, и обеспечения пожарной безопасности.
        \end{definition}
        \subsection{оповещение начеления города Москвы о ЧС.}
        \begin{definition}
            Оповещение населения города Москвы о ЧС -- это доведение до населения города Москвы единого сигнала ГО и экстренной информации об опасностях, возникающих при угрозе возникновения или возникновении ЧС природного или техногенного характера, а также при введении военных действий или вследствие этих действий, о правилах поведения населения и необходимости прорведения мероприятий по защите (далее -- сигнал ГО и экстренная информация о ЧС).
        \end{definition}
        При проведении оповещения населения города Москвы о чрезвычайных ситуациях задействуются:
        \begin{itemize}
            \item Сеть электросиренного оповещения
            \item Линии уличной звукофикации
            \item Сеть УКВ-ЧМ (радиовещания)
            \item Сети телевещания (каналы звукового сопровождения)
            \item Сети кабельного телевидения города Москвы
            \item Сети подвижной радиотелефонной связи
            \item Телефонная сеть связи города Москвы
            \item Локальные и объектовые системы оповещения
            \item Территориальные эелементы Общероссийской комплексной системы информирования и оповещение населения в местах массового пребывания людей (ОКСИОН)
            \item Иные технологические элементы системы оповещения населения, введенные в эксплуатацию, радиотрансляционная сеть города Москвы (сеть проводного радиовещания), в том числе средства наружной рекламы и информации, а также электронные дисплеи, рассположенные на территории и объектах города Москвы.
        \end{itemize}
        \subsection{Порядок действий по единному сигналу оповещения ГО}
        Если сигнал застал вас на работе или дома:
        \begin{enumerate}
            \item Включить радио, радиотрансляционные и телевизионные приборы.
            \item Внимательно прослушать сообщение о сложившейся ситуации и порядке действий. \textbf{Оцените обстановку! Примите решение!} 
            \item Действовать в соответствии с переданным сообщением и вашим решением.
            \item Доведите информацию до своих родных, близких, не забудьте о пожилых соседях.
        \end{enumerate}
        Если сигнал застал вас на улице:
        \begin{enumerate}
            \item Прослушать сообщение, передаваемое уличными громкоговорителями и подвижными средствами оповещения.
            \item Прочитать информационное сообщение на уличных светодиодных экранах, плазменных панелях, распооженных в местах массового пребывания людей.\textbf{Оцените обстановку!}
            \item Действоватьв соответствии с переданным сообщением и вашим решением. 
        \end{enumerate}
        \begin{center}
            \textbf{Телефоны вызова экстренных служб г. Москвы} 
        \end{center}
        11 февраля 2013 гола президентом РФ подписан ФЗ РФ N 9-ФЗ, устанавливающий \textbf{номер "112"} единным номером вызова экстренных оперативных служб для приема сообщений о пожарах и ЧС в России в телефонных сетях местной телефонной связи.\\
        Службы:
        \begin{itemize}
            \item Служба пожарной охраны: 101
            \item МВД: 102
            \item Служба медицинской скорой помощи: 103
            \item Аварийная служба газовой сети: 104
        \end{itemize}
        Телефоны вызова экстренных служб г. Москвы:
        \begin{itemize}
            \item Антитеррор: 8-495-914-22-22
            \item Центр управления кризисными ситуациями ГУВД г. Москвы: 8-495-995-99-99
            \item Центр управления кризисными ситуациями Упаравления по ЗАО Департамента ГОЧСиПБ г. Москвы: 8-495-415-29-11
            \item Экстренные службы МГУ: 8-495-939-37-50
            \item Дежурная часть ОВД МГУ: 8-495-939-28-32
            \item Оперативный дежурный по МГУ: 8-495-939-11-11
            \item Главная диспетчерская МГУ: 8-495-939-26-07
        \end{itemize}
        \subsection{Стадии развития ЧС.}
        ЧС любово типа в своем развитии, как правило, проходят 4 стадии (фазы).\\
        \textbf{Первая} -- стадия накопления отклонений от нормального состояния или процесса. Иными словами, это сдадия зарождения ЧС, которая может длиться сутки, месяцы, годы.\\
        \textbf{Вторая} -- инициирование чрезвычайного события, лежащего в основе ЧС.\\
        \textbf{Третья} -- процесс чрезвычайного события, во время которого проходит высвобождение факторов риска (энергии или вещества), оказывающих неблагоприятное воздествие на население, объекты и природную среду.\\
        \textbf{Четвертая} -- стадия затухания (действие отстаточных факторов и сложившихся чрезвычайных условий), которая хронологически охватывает периол перекрытия (ограничения) источника опасности -- локализации ЧС, до полной ликвидации ее прямых и косыенных последствий, включая всю цепочку вторичных последствий. Эта фаза при некоторых ЧС может по времени начинаться еще до завершения третьей фазы. Продолжительность этой стадии может составлять годы, а то и десятилетия. 
        \begin{definition}
            Авария -- это проишествие, происходящее по техногенным причинам, а также из-за случайных внешних воздействий и приводящих к разрушению технических устройств и сооружений.
        \end{definition}
        \begin{definition}
            Катастрофы -- крупная авария повлекшая за собой человеческие жертвы, ущерб здоровью людей либо разрушения, либо уничтожение объектов, материальных ценностей в значительных размерах, а также приведшая к серьезному ущербу ОКС.
        \end{definition}
        \subsection{Классификация ЧС.}
        По сфере возникновения: 
        \begin{itemize}
            \item Техногенного характера
            \item Природного характера 
            \item Биолого-социального характера
            \item Экологического характера
        \end{itemize} 
        По характеру протекания:
        \begin{itemize}
            \item Взрывного характера
            \item Плавно протекающие
        \end{itemize}
        Классификация ЧС по масштабу и степени ущерба -- размер материального ущерба (размер ущерба ОКС и материальных потерь.) (Постановление Правительства РФ N 304 от 21.05.2007 г. в ред. от 20.12.2019 г.)
        \begin{itemize}
            \item Локальные
            \item Местные
            \item Территориальные
            \item Региональные
            \item Федеральные и трансграничные.
        \end{itemize}
        Классификация по ведомственной принадлежности:
        \begin{itemize}
            \item В строительстве
            \item Промышленности
            \item Транспортной
            \item Коммунально-бытовой сфере
        \end{itemize}
        \subsection{Виды ЧС техногенного характера.}
        Техногенные ЧС могут возникать на основе событий техногенного характера вследствие констрективных недостатков объекта (сооружения, косплекса, низкой квалификации персонала, нарушения техники безопасности в ходе эксплуатации объекта и др.). ЧС техногенного характера могут протекать с загрязнением или без загрязнения ОКС.
        \begin{itemize}
            \item Аварии с выбросом (угорозой выброса) химически опасных веществ (ХОВ): аварии с выбросом (угрозой выброс) ХОВ при их производстве, переработке или хранении (захоронении); утрата источников ХОВ; аварии с химическими боеприпасами и тд.
            \item Аварии с выбросом (угрозой выброса) радиоактивных веществ: аварии на атомных станциях; аварии транспортных средств и космических аппаратовс ядерными установками; аварии с ядернвми боеприпасами в местах их хранения, эксплуатации или установки; утрата радиоактивных источников и др.
            \item Аварии с выбросом (угрозой выброса) биологически опасных веществ\\ (БОВ): аварии с выбросом (угрозой выброса) биологически опасных веществ на предсприятиях и в научно-исследовательских учреждениях; утрата БОВ и др.
            \item Аварии на электроэнергетических системах: аварии на автономных электростанциях с долговременным перерывом всех потребителей; выход из строя транспортных электроконтактных сетей и др.
            \item Аварии на коммунальных системах жизнеобеспечения: аварии на кализационных системх с массовым выбросом загрязняющих веществ; аварии на тепловых сетях в холодное время года; авария в системах снабжения населения питьевой водой; аварии на коммунальных газопроводах.
            \item Аварии на очистных сооружениях: аварии на очистных сооружениях сточных вод промышленных мероприятий с массовый выбросом загрязняющих веществ;аварии на очистных сооружениях промышленных газов с массовым выбросом загрязняющих веществ.
        \end{itemize} 
        \begin{definition}
            ЧС социального характера -- обстановка на определенной территории, сложившаяся в результате возникновения опасных противоречий и конфликтов в сфере социальных отношений, которые могут повлечь или повлекли зп собой человеческие жертвы, ущерб здоровью людей или ОКС, значительные материальные потери или нарушение условий жизнедеятельности людей.
        \end{definition}
        \subsection{«Единная государственная сисетема предупреждения и ликвидации ЧС (РСЧС). ГО РФ.»}
        Сл 8 Гражданская оборона - система мероприятий по подготовке к защите и по защите населения, материальных и культурных ценностей на территории Российской Федерации от опасностей, возникающих при военных конфликтах или вследствие этих конфликтов, а также при ЧС природного и техногенного характера;

        ФЗ "О гражданской обороне" от 12.02.1998 N 28-ФЗ ( ред. 04.08.2023 N 440-ФЗ).

        Сл11-12-13  Задачи ГО : подготовка населения в области гражданской обороны; -оповещение населения об опасностях, возникающих при военных конфликтах или вследствие этих конфликтах, а также при возникновении чрезвычайных ситуаций природного и техногенного характера;

        -эвакуация населения, материальных и культурных ценностей в безопасные районы;

         -предоставление населению  средств индивидуальной и коллективной защиты;

        -проведение мероприятий по световой маскировке и другим видам маскировки;

        -проведение аварийно-спасательных и других неотложный работ при военных конфликтах или вследствие этих конфликтов, а также при ЧС природного и техногенного характера;
        
        -первоочередное жизнеобеспечение населения, пострадавшего при военных конфликтах или вследствие этих конфликтов, а также при ЧС природного и техногенного характера;

        - борьба с пожарами, возникшими при военных конфликтах 

        - обнаружение и обозначение районов, подвергшихся радиоактивному, химическому, биологическому или иному заражению;
        
        - санитарная обработка населения, обеззараживание зданий и сооружений, специальная обработка техники и территорий;
        СЛ14- СТРУКТУРА ГО:  1.ОРГАНЫ УПРАВЛЕНИЯ, 2.СИЛЫ  и СРЕДСТВА, 

        3. ЗАПАСЫ МТС,ПРОД-Я и др. СРЕДСТВ, 4.Система управления, оповещения и связи.

        СЛ -15 Органы управления по делам гражданской обороны
        Постоянно действующими органами управления гражданской обороны являются:

        -на федеральном уровне — МЧС России, а также образованные для решения задач в области защиты населения и территорий от ЧС подразделения ФОИВ и государственных корпораций;

        -на муниципальном уровне —создаваемые при органах местного самоуправления органы, специально уполномоченные на решение задач в области защиты населения и территорий от ЧС;
        
        -на объектовом уровне структурные подразделения организаций, специально уполномоченные на решение задач в области защиты населения и территорий от ЧС
        КООРДИНИРУЮЩИЕ ОРГАНЫ управления
        
        -на федеральном уровне Правительственная комиссия по Предупреждению и Ликвидации  ЧС и обеспечению Пожарной Безопасности ( КЧС  ПБ )
        
        -  на муниципальном уровне     КЧС  ПБ   ОГАНОВ ИСПОЛНИТЕЛЬНОЙ ВЛАСТИ МУН. ОБРАЗОВАНИЯ;
        
        -на объектовом - КЧС ПБ организаций, в полномочия которых входит решение вопросов по защите населения и территорий от ЧС, в том числе по обеспечению безопасности людей на водных объектах.
        ОРГАНЫ Повседневного управления
        
        - Федеральный уровень-НАЦИОНАЛЬНЫЙ центр управления в кризисных ситуациях МЧС,
        
        - Муниципальный -- единые дежурно-диспетчерские службы (ЕДДС) муниципальных образований, подведомственные органам местного самоуправления, ДДС экстренных оперативных служб, а также другие организации (подразделения), обеспечивающие деятельность органов местного самоуправления в области ЗНиТ от ЧС, предназначенными и привлекаемыми для предупреждения и ликвидации ЧС, осуществления обмена информацией и оповещения населения о ЧС;

        на Объектовом уровне - подразделения организаций, обеспечивающие их деятельность в области ЗНиТ от ЧС, управления силами и средствами, предназначенными и привлекаемыми для предупреждения и ликвидации ЧС, осуществления обмена информацией и оповещения населения о ЧС

        СЛ 19 В законе определены полномочия органов власти и организаций

        Статья 9. Полномочия организаций в области гражданской обороны:       

        1.планируют и организуют проведение мероприятий по ГО
        
        2.проводят мероприятия по поддержанию своего устойчивого функционирования в военное время;
        
        3.осуществляют подготовку своих работников в области ГО;
        
        4.создают и содержат в целях ГО запасы материально-технических, продовольственных, медицинских и иных средств.
        Организации, отнесённые в установленном порядке к категориям по ГО, создают и поддерживают в состоянии готовности нештатные формирования по обеспечению выполнения мероприятий по ГО.
        
        Сл 20. Статья 10. Права и обязанности граждан Российской Федерации в области ГО
        проходят подготовку в области ГО;
        принимают участие в проведении других мероприятий по ГО;
        оказывают содействие органам государственной власти и организациям в решении задач в области ГО.
        
        СЛ 21 (ПОЛОЖЕНИЕ «ОБ ОРГАНИЗАЦИИ И ВЕДЕНИИ ГО г.МОСКВЕ» В ред. от 15.02.2022 N 211-ПП)
        Настоящее Положение разработано в соответствии с Федеральным законом от 12 февраля 1998 г. N 28-ФЗ "О гражданской обороне", постановлением ППРФ от 26 ноября 2007 г. N 804 "Об утверждении Положения о ГО в РФ" и определяет организационные основы ГО, содержание основных мероприятий гражданской обороны, состав сил и средств ГО, порядок организации и ведения ГО в городе Москве.
        
        1.3. Подготовка к ведению ГО заключается в заблаговременном выполнении мероприятий по подготовке к защите населения, материальных и культурных ценностей на территории города Москвы от опасностей, возникающих при военных конфликтах или вследствие этих конфликтов, а также при возникновении чрезвычайных ситуаций природного и техногенного характера.
        
        1.4. Ведение ГО заключается в выполнении мероприятий по защите населения, материальных и культурных ценностей на территории города Москвы от опасностей, возникающих при военных конфликтах или вследствие этих конфликтов, а также при возникновении чрезвычайных ситуаций природного 
        и техногенного характера.
        
        Органы государственной власти города Москвы, органы местного самоуправления городских округов и поселений в городе Москве (далее - органы местного самоуправления) и организации, расположенные на территории города Москвы (далее - организации), в целях решения задач в области ГО в соответствии с установленными полномочиями создают и содержат силы, средства, объекты гражданской обороны, запасы материально-технических, продовольственных, медицинских и иных средств, планируют и осуществляют мероприятия по гражданской обороне.

        СЛ36                     Нормативно-правовая основа создания РСЧС
        Цели  Федерального закона №68-ФЗ
        
        -предупреждение возникновения и развития чрезвычайных ситуаций;
        
        -снижение размеров ущерба и потерь от ЧС;
        
        -разграничение полномочий в области защиты населения и территорий от ЧС между ФОИВ, ОИВ субъектов РФ, органами местного самоуправления и организациями
        
        СЛ 40 Статья 19. Обязанности граждан Российской Федерации в области защиты населения и территорий от чрезвычайных ситуаций
        соблюдать законы и иные нормативные правовые акты РФ, законы и иные нормативные правовые акты субъектов РФ в области ЗНТ от ЧС;
        
        соблюдать меры безопасности в быту и повседневной трудовой деятельности, не допускать нарушений производственной и технологической дисциплины, требований экологической безопасности, которые могут привести к возникновению ЧС;
        изучать основные способы защиты населения от ЧС, приемы оказания первой помощи пострадавшим, правила охраны жизни людей на водных объектах, правила пользования коллективными и индивидуальными СЗ, постоянно совершенствовать свои знания и практические навыки в указанной области;
        выполнять установленные в соответствии с настоящим ФЗ правила поведения при введении режима повышенной готовности или ЧС;
        эвакуироваться с территории, на которой существует угроза возникновения ЧС, или из зоны ЧС при получении информации о проведении эвак. мер-й .
        
        ПРАВА Ст.18…. - быть информированными о риске, которому они могут подвергнуться в определенных местах пребывания на территории страны, и о мерах необходимой безопасности;
        -обращаться лично, а также направлять в государственные органы и органы местного самоуправления индивидуальные и коллективные обращения по вопросам ЗНиТ от ЧС, в том числе обеспечения безопасности людей на водных объектах;
        
        СЛ 41     ПОЛОЖЕНИЕ о подготовке граждан Российской Федерации, иностранных граждан и лиц без гражданства в области защиты от чрезвычайных ситуаций природного и техногенного характера (ПП РФ № . № 1485 от 2020г.)
        
        СЛ 61 Режимы функционирования РСЧС
        Режим функционирования органов управления и сил РСЧС - это определяемые в зависимости от обстановки, прогнозирования угрозы  и возникновения ЧС порядок организации деятельности органов управления и сил единой государственной системы предупреждения и ликвидации ЧС и основные мероприятия, проводимые указанными органами и силами.
        Повседневной деятельности  -при отсутствии угрозы возникновения ЧС;
        Повышенной готовности при угрозе возникновения ЧС;
        Чрезвычайной ситуации при возникновении и ликвидации ЧС.
        
        СЛ65    Обязанности организаций в области защиты населения и территорий от ЧС.
        
        - а) планировать и осуществлять необходимые меры в области защиты работников организаций и подведомственных объектов производственного и социального назначения от  ЧС;
        
         б) планировать и проводить мероприятия по повышению устойчивости\\ функционирования организаций и обеспечению жизнедеятельности работников организаций в ЧС;
        
        в) обеспечивать создание, подготовку и поддержание в готовности к применению сил и средств предупреждения и ликвидации ЧС, осуществлять обучение работников организаций способам защиты и действиям в ЧС;
        
        г) создавать и поддерживать в постоянной готовности локальные системы оповещения о ЧС;
        
         д) обеспечивать организацию и проведение аварийно-спасательных и других неотложных работ на подведомственных объектах производственного и социального назначения и на прилегающих к ним территориях в соответствии с планами предупреждения и ликвидации ЧС;
        
        е) финансировать мероприятия по защите работников организаций и подведомственных объектов производственного и социального назначения от чрезвычайных ситуаций;
        
        ж) создавать резервы финансовых и материальных ресурсов для ликвидации чрезвычайных ситуаций;  (ст.14, 68-ФЗ)

        \subsection{Формы подготовки в области ГО (по группам лиц, подлежщих подготовке) ППРФ 841 в ред. 23 г.}
        \textbf{Работающее население:}\\
        (в ред. Постановления Правительства РФ от 15.08.2006 N 501)
        (см. текст в предыдущей редакции)
        
        а) утратил силу с 1 сентября 2023 года. 
        - Постановление Правительства РФ от 21.01.2023 N 51;
        (см. текст в предыдущей редакции)
        а(1) прохождение вводного инструктажа по гражданской обороне по месту работы;
        (пп. "а(1)" введен Постановлением Правительства РФ от 19.04.2017 N 470)
        
        б) участие в учениях, тренировках и других плановых мероприятиях по гражданской обороне, в том числе посещение консультаций, лекций, демонстраций учебных фильмов;
        (в ред. Постановления Правительства РФ от 30.09.2019 N 1274)
        (см. текст в предыдущей редакции)
        
        в) самостоятельное изучение способов защиты от опасностей, возникающих при военных конфликтах или вследствие этих конфликтов.

        \textbf{Обучающиеся:}\\
        С 01.09.2024 в п. "а" п. 5 вносятся изменения (Постановление Правительства РФ от 04.11.2023 N 1859).

        а) обучение (в учебное время) по предмету "Основы безопасности жизнедеятельности" и дисциплине "Безопасность жизнедеятельности";

        (в ред. Постановления Правительства РФ от 19.04.2017 N 470)

        б) участие в учениях и тренировках по гражданской обороне;

        в) чтение памяток, листовок и пособий, прослушивание радиопередач и просмотр телепрограмм по тематике гражданской обороны.

        \textbf{Неработающее население (по месту жительства):}

        а) посещение мероприятий, проводимых по тематике гражданской обороны (беседы, лекции, вечера вопросов и ответов, консультации, показ учебных фильмов и др.);

        б) участие в учениях по гражданской обороне;

        в) чтение памяток, листовок и пособий, прослушивание радиопередач и просмотр телепрограмм по тематике гражданской обороны.

        \section{Защита населения и территорий при авариях на химически опасных объектах с выбросом (проливом) аварийно химически опасных веществ в окружающую среду.}

        \begin{center}
            \textbf{Аварийно химически опасные вещества.} 
        \end{center}
        \begin{definition}
            Опасное химическое вещество (ОХВ) -- химическое вещество, прямое или опосредственное воздействие которого может вызвать острые хронические заболевания людей и их гибель.
        \end{definition}

        \begin{definition}
            Химическое заражение среды -- распространение ОХВ в ОКС, в концентрациях или кодичествах, создающих угрозу для людей, с/х животных и растений в течение опеределенного времени.
        \end{definition}

        \subsection{Классификация опасных химических веществ (ОХВ)}
        \begin{definition}
            Аварийно-химически опасные вещества (АХОВ) -- это опасное вещество, используемые в экономике, при аварийном ывбросе (разливе) которого может произойти заражение ОКС в поражающий живой организм концентрациях (токсодозах) (ГОСТ Р22.9.05-95).
        \end{definition}
        \begin{definition}
            Постоянно действующие химически опасные вещества\\ (ПД ХОВ) -- систематически оказывающие вредное воздействие на организм человека.
        \end{definition}
        \begin{definition}
            Бытовые химически опасные вещества (БХОВ) -- способные вызвать поражение населения при их боевом применении возможным противником или при авариях на объектах их временного хранения и на предприятиях по уничтожению.
        \end{definition}
        \begin{center}
            \textbf{Классификация АХОВ} 
        \end{center}
        По классу опасности (степень воздействия на человека):
        \begin{itemize}
            \item Чрезвычайно опасные (1кл) (менее 500 мг/м.куб)
            \item Высоко опасные (2кл) (500-5000 мг/м.куб)
            \item Умеренно опасные (3кл) (5001-50000 мг/м.куб)
            \item Опасные (4кл) (более 50000 мг/м.куб)
        \end{itemize}
        По характеру воздействия на организм человека:
        \begin{itemize}
            \item Раздражающие -- хлор, сернистый ангидрид, хлор пикрин.
            \item удушающего действия -- хлорпикрин, фосген.
            \item Прижигающего действия -- аммиак, соляная кислота.
            \item Психогенного действия -- формальдегид, бромистый и хлористый метил.
            \item Метаболические яды -- оксид этилена, дихлорэтан.
        \end{itemize}
        Физико-химические свойства АХОВ:
        \begin{itemize}
            \item Цыет, запах
            \item Агрегатное состояние, летучесть
            \item Температура кипения
            \item Плотность
            \item Растворимость
            \item Токсичность
        \end{itemize}
        \begin{definition}
            Химически опасный объект (ХОО) -- объект, на котором хранят, перерабатывают, используют или транспортируют опасные химические вещества, при аварии на котором или при разрушении которого может произойти гиболь или химическое поражение людей, с/х животных и растений, а также химическое заражение ОКС.
        \end{definition}
        ХОО могут быть классифицированы по следующим показателям:
        \begin{enumerate}
            \item По сфере использования
            \item по способам и условиям хранения
            \item По категории химической опасности
        \end{enumerate}
        \begin{definition}
            Под химической аварией понимается авария на химически\\ опасном объекте, сопровождабщаяся проливом или выбросом АХОВ, способоная привести к гибели или химическому заражению людей, продовольствия, пищевого сырья и кормов, с/х животных и растений или к химическому заражению ОКС.
        \end{definition}
        Классификация аварий на ХОО:
        \begin{itemize}
            \item По месту возникновения
            \item По причинам возникновения
        \end{itemize}
        \begin{center}
            \textbf{Характер воздействия химического заражения на население.} 
        \end{center}
        АХОВ оказывают химическое воздействие на ферменты организма (химические и биологические вещества, играющие важную роль в обмене веществ как внутри организма, так и между ним и внешней средой), приводящее к торможению или прекращению ряда важнейших функций организма и его поражению в различной степени. Наиболее часто отравления АХОВ происходят в результате ингаляционного поступления его в организм человека.

        \subsection{Действие населения при химичсеком заражении местности.}

        Независимо от того, где вы находитесь (на работе, на производстве, дома), ощутив резкий или непривычный запах, немедленно сообщите об этом либо должностным лицам предприятия, компетентным органам либо, получив информацию о выбросе в атмосферу химически опасных веществ и об опасности химического заражения, знайте, что поражающее действие конкретной ядовитого вещества на человека зависит от ее концентрации в воздухе и продолжительности, поэтому если нет возможности покинуть опасную зону, не паникуйте и принимайте САМОСТОЯТЕЛЬНО  меры безопасности;

        ЕСЛИ СИГНАЛ (информация) ЗАСТАЛ ВАС НА УЛИЦЕ :
        \begin{enumerate}
            \item Защитите органы дыхания простейшими СИЗ ОД (платок, шарф и тд).
            
            (ПОМНИТЕ, что поражающее действие конкретной ядовитого вещества на человека зависит от ее концентрации в воздухе и продолжительности, поэтому если нет возможности покинуть опасную зону, не паникуйте и продолжайте принимать меры безопасности);
            \item Необходимо сориентироваться, где находится источник опасности; 
            \item Начать ускоренное движение в сторону, перпендикулярную направлению ветра; · если на пути движения встретятся препятствия (высокий забор, река и т.п.), не позволяющие быстро выйти из опасной зоны, а поблизости находится жилое или общественного назначения здание, необходимо временно укрыться в нём.
            \item Если дом рядом.
            
            \begin{itemize}
                \item Пред тем как войти в квартиру снимите верхнюю одежду, оставьте ее на площадке; соседи...
                \item Наденьте СИЗ ОД, включите канал  телевидения или слушайте голосовые сообщения, передаваемые с помощью громкоговорителей; и одновременно, усилить ГЕМЕТИЗАЦИЮ квартиры ЗАКРОЙТЕ окна, произведите  экстренную герметизацию квартиры, вентиляционные отверстия, дымоходы, уплотните щели в окнах и на стыках рам. Не забудте верхнюю одежду на площадке
                - упакуйте свою верхнюю одежду, оставленную на площадке, в  полиэтилен; примите душ, дружите с СИЗ ОД.
                \item оцените обстановку , примите РЕШЕНИЕ! (цель и порядок действий)!
                
                Для этого заклейте или заделайте подручными средствами (лейкопластырь, скотч, обычная бумага) щели в оконных рамах, дверях, навесьте на дверные коробки плотную ткань (одеяло), предварительно смочив водой, вентиляционные отверстия прикройте бумагой, полиэтиленовой пленкой, клеенкой;

                Помните! Надежная герметизация жилища значительно уменьшает возможность проникновения химически опасных веществ в помещение.
                \item Выполнив работу по усилению герметизации квартиры, попытайтесь связаться с родными и близкими.
                \item         4.6 Все укрывшиеся в зданиях должны быть готовы к выходу из зоны заражения по указанию органов ГОЧС. Проводим мероприятия по подготовке к ЭВАКУАЦИ:
                - готовим СИЗ ОД, средства защиты кожи (обувь, перчатки, защитные очки, головной убор, верх. одежду, (согласно времени года), белье, туалетные принадлежности (возможно будут проводить  ПОЛНУЮ Санитарную Обработку), лекарства документы, денежные средства, телефон с доп. питанием, фонарь, продукты питания, документы, денежные ср-ва. Не забывайте о своей грузоподъемности!
                - инспектируем холодильник с продуктами, готовим мусор для эвакуации.
                \item С получением, информации об ЭВАКУАЦИИ, перекрываем газ,  выключаем электроснабжение квартиры, делаем попытку сообщить обстановку родным, одеваем простейшие СИЗ К и ОД(возможен вариант получения более надежного СИЗ ОД на пункте выдачи, организованного силами муниципальных органов власти).   забираем мусор, дублируем информацию соседям и (если  вы за ранее побеспокоились и знаете адрес)   направляемся в район сбора для эвакуации. 
                \item Если Вы не успели выполнить мероприятия по подготовке к эвакуации и прошла информация ОТБОЙ!!! 
                После получения сигнала “Отбой химической тревоги”, не спешите открыть окна для проветривания помещения и 
                 выбрасывать СИЗ ОД.  Проведите влажную уборку, успокойтесь, распакуйте все вещи, разложите по своим местам, восстановите контакт с родными, проветрите квартиру. 
            \end{itemize}

            
        \end{enumerate}
        

        \section{Здоровье.}
        \begin{definition}
            Здоровье -- это плод воздействия разнообразных факторов на уникальнцю систему "человек". Каждый человек должен уметь управлять своим здоровьем с учетом риска заболеваний и особенностей психологического, физиологического факторов.
        \end{definition}
        Для самостоятельного управления здоровьем каждому необходимо уметь создать комфорт души и тела, что достигается определенным образом жизни, отношением к средствам оздоровления и исцеления от недугов.
        \begin{definition}
            Здоровье -- это состояние полного физического, духовного и социального благополучия, а не только отсутствие заболеваний или физических дефектов.
        \end{definition}
        Здоровье человека зависит от образа жизни, поведения, образа мыслей.
        \subsection{Самые частые ошибки при оказании первой помощи.}
        Это один из главных принципов любой медицинской помощи, в том числе и первой.

        Той, которую оказывают по мере  возможности и умения люди, случайно оказавшиеся рядом с пострадавшим или больным человеком, пока не прибыла Скорая помощь.

        Для этого и учат обычно простейшим приемам оказания первой помощи — в школе на уроках ОБЖ, во время учений по гражданской обороне или инструктажа по технике безопасности при приеме на работу. «Мы все учились понемногу когда-нибудь и как-нибудь», в том числе и навыкам оказания помощи. Вот только полученные «как-нибудь» знания, не необходимые в повседневной жизни, обладают свойством стираться, забываться и искажаться. Это нормальное явление — так уж устроена наша память.

        И когда вдруг случается что-то, что требует немедленных и эффективных действий, все полученные когда-то знания извлекаются из памяти в виде сумбурного набора штампов: наложение жгута при кровотечении, шина при переломе, непрямой массаж сердца и искусственное дыхание. Ах да, еще что-то делают при судорогах. И при ожогах то ли нужно чем-то смазать, то ли наоборот… Не делать лишнего

        Если бы вы знали, сколько людей остались инвалидами из-за того, что их не оставили там, где они находились при несчастном случае до приезда специалистов… Существуют травмы, при которых неосторожное движение может нанести непоправимый вред пострадавшему — например, при повреждении позвоночника или серьезных переломах. Но очень многие искренне считают необходимым вытащить жертву ДТП из смятого авто, чтобы уложить «поудобнее». Или перевернуть «поудобнее» упавшего с высоты.

        Врачи и спасатели очень просят: не двигайте пострадавшего без крайней необходимости и непосредственной угрозы его жизни.
 
        Упавшего с высоты лучше вообще не трогать, если он упал не в воду и не в огонь. 

        Жертву ДТП извлекают из автомобиля, не дожидаясь спасателей, только если машина может загореться, утонуть или упасть вниз.

        Можно и нужно: осторожно остановить кровотечение и быть рядом с пострадавшим, оказывая ему моральную поддержку до приезда специалистов, разговаривая, успокаивая и подбадривая его, если он в сознании.Не лезть на рожон
        
        Поэтому очень важно накрепко запомнить следующее: прежде чем бросаться на помощь, необходимо оценить степень опасности для того, кто собирается помочь. Иначе количество жертв только увеличится, а вызвать специалистов будет уже некому.

        Став свидетелем происшествия или обнаружив пострадавшего, сначала нужно вызвать спасателей и врачей. 

        Потом оценить ситуацию и свои возможности. И только потом бросаться на помощь, стараясь делать это эффективно и соблюдением принципа, описанного выше — «Не навреди».

        Например, заметив пострадавшего от поражения электротоком, необходимо сначала убедиться, что вы не попадете под удар сами: отключить рубильник, отбросить в сторону с помощью предмета, не проводящего электричество, оголенный провод.

        Если человек упал в воду с моста, не прыгайте за ним, даже если вы мастер спорта по прыжкам в воду. Вы не знаете глубины и того, что находится под водой, а значит, можете стать второй жертвой.

        \subsection{Первая помощь при электроударе.}
        Сперва всегда нужно оценить обстановку, после чего:
        \begin{enumerate}
            \item Обеспечить собственную безопасность
            \item Прекратить (обесточить) действие электрического тока
            \item Вызвать скорую помощь
            \item Оказать первую помощь: в критической ситуации промыть проточной водой $t=12-18$ гр. $15-20$ мин., влажной тканью накрываем место ожога и ждем помощи специалистов, даем обезболивающее средство (ибупрофен).
        \end{enumerate}
        \subsection{Необходимость скорой помощи.}

        Невнятная речь, трудности со сгибанием шеи, коричневая рвота и другие признаки опасных болезней, которые нужно срочно лечить.

        \textbf{1.Впервые появилась острая боль в груди:}

        Далеко не факт, что это именно сердечный приступ, но лучше перестраховаться. Ощущения в грудной клетке варьируются от стеснения и давления до острой жгучей боли, которая может распространяться на левую руку, лопатку, нижнюю челюсть. При этом таблетки от боли в сердце не помогают. Все это сопровождается резкой слабостью, головокружением, одышкой, холодным и липким потом.

        Иногда сердечный приступ проявляется болью в верхней части живота или ощущениями по типу изжоги. 

        \textbf{2 Впервые появились перебои в сердце:}

        Это указывает на аритмию — нарушение нормального ритма сердца. Вы можете ощущать, что оно бьется чаще, реже или нерегулярно, как бы с остановками. 
        
        Опасные аритмии не обязательно проявляются заметными признаками, также и выраженные симптомы не всегда говорят о тяжести проблемы. Но если это состояние появилось впервые, сопровождается слабостью, головокружением, одышкой, потерей сознания — лучше сразу обращайтесь в скорую помощь. Если аритмию не лечить, то она может вызвать сердечную недостаточность, тромбоз, повышает риск инфаркта, а иногда вызывает остановку сердца. 

        \textbf{3.Тяжелая аллергическая реакция (анафилаксия):}

        Она возникает внезапно, состояние ухудшается очень быстро. Причиной может быть еда, лекарства, укусы насекомых.
        
        \begin{itemize}
            \item  Симптомы включают: головокружение, резкую слабость;
            \item Одышку;
            \item Kожа покрывается липким потом;
            \item Cпутанность сознания и беспокойство;
            \item Обморок;
        \end{itemize}

        Могут быть и другие симптомы аллергии, в том числе зуд, сыпь (крапивница), тошнота, отек лица и шеи (ангионевротический отек) или боль в животе.

        Тяжелая аллергическая реакция быстро переходит в смертельно опасный анафилактический шок. Поэтому, если вы заметили похожие симптомы, вызывайте скорую помощь.

        \textbf{4. Острая внезапная боль в животе:}
        
        Каждый из нас когда-либо ощущал боль в животе из-за пищевого отравления, несварения или инфекции. Это неприятно, но такая боль терпима и проходит сама или с помощью безрецептурных лекарств.
        
        Но если боль появилась внезапно, или она настолько сильная, что вы буквально не находите себе места — это признак катастрофы в брюшной полости, например прободения язвы, острого панкреатита, аппендицита или инфаркта брюшных артерий.
        
        В этом случае, а также если сильная боль в животе сопровождается неукротимой тошнотой и рвотой, лихорадкой, пожелтением кожи или головокружением, нужно вызывать скорую помощь.

        \textbf{5. Кашель с кровавой мокротой:}

        Вы можете кашлять небольшим количеством ярко-красной крови или пенистой мокротой с кровавыми прожилками. Врачи называют это кровохарканьем — обычно оно говорит о повреждении в легких.
    
        Если кровь темная и содержит кусочки пищи или что-то похожее на кофейную гущу, скорее всего, она поступает из пищеварительного тракта. 

        Чаще всего кровохарканье возникает из-за инфекции в легких, например плеврита или пневмонии. Также это может быть признаком рака, туберкулеза или отека легких.

        Если крови в мокроте мало и вы чувствуете себя хорошо, запишитесь на прием к врачу. Но если крови много, она не останавливается, при этом у вас слабость или одышка — звоните в скорую помощь. 

        \textbf{6. Рвота кровью:}
        
        Она возникает из-за кровотечения в верхней части пищеварительного тракта: в пищеводе, желудке или в двенадцатиперстной кишке. Причиной может быть язва или разрыв вен. Кровь в рвоте бывает красной, коричневой или почти черной — цвета кофейной гущи. Окрас зависит от места повреждения: чем оно выше, тем более яркой будет кровь.

        Изредка кровавая рвота возникает, когда в желудок попадает кровь из носа или рта — это не так опасно, как желудочно-кишечное кровотечение, которое может вызвать шок.

        Вызывайте скорую помощь, если крови много и есть признаки шока:
        \begin{itemize}
            \item Учащенное, неглубокое дыхание.
            \item Головокружение, сильная слабость.
            \item Затуманенное зрение.
            \item Спутанность сознания, обморок.
            \item Холодная, липкая, бледная кожа.
        \end{itemize}

        \textbf{7. Внезапная резкая боль в яичках:}
        
        Она может возникнуть при перекруте яичка. По неясным причинам семенной канатик, на котором висит яичко, перекручивается, из-за чего пережимаются кровеносные сосуды. Если через несколько часов не вернуть яичко на место, то оно погибнет. 
    
        Помимо боли бывают и другие симптомы: тошнота и рвота, лихорадка, отек мошонки. Чаще всего эта проблема возникает в 12–18 лет, но может быть в любом возрасте, даже у новорожденных.

        \textbf{8. Нарушения речи:}
        
        Она может возникнуть при перекруте яичка. По неясным причинам семенной канатик, на котором висит яичко, перекручивается, из-за чего пережимаются кровеносные сосуды. Если через несколько часов не вернуть яичко на место, то оно погибнет. 
    
        Помимо боли бывают и другие симптомы: тошнота и рвота, лихорадка, отек мошонки. Чаще всего эта проблема возникает в 12–18 лет, но может быть в любом возрасте, даже у новорожденных.

        \textbf{8. Нарушения речи:}
        
        Сюда относятся любые странности в разговоре. Человек может говорить неразборчиво, путает слова, не понимает собеседника или отвечает невпопад. Иногда речь пропадает полностью. Насторожить должна и потеря способности читать или писать. Все это указывает на нарушения в головном мозге, чаще всего вызванные инсультом. 

        \textbf{9. Трудно сгибать шею на фоне высокой температуры:}
        
        Это признак менингита — инфекционного воспаления оболочки мозга, которое может вызывать потерю слуха, трудности с памятью и обучением, и даже смерть. 

        Помимо высокой температуры и невозможности согнуть шею, менингит проявляется очень сильной головной болью, чувствительностью к свету, тошнотой, рвотой, иногда судорогами и кожной сыпью. В тяжелых случаях смерть наступает в течение нескольких часов.

        \textbf{10. Внезапное ухудшение зрения:}
        
        Потеря зрения считается внезапной, если она развивается в течение нескольких минут, часов или пары дней. Вы можете перестать видеть полностью или частично, на один, оба глаза, или на какую-то часть поля зрения. Сюда же относится появление тумана или темных пятен перед глазами. 

        Чаще всего это случается из-за инсульта, отслоения сетчатки, закупорки сосудов сетчатки тромбом и травмы. 

        В таких случаях у вас очень мало времени, чтобы определить диагноз и начать лечение, в противном случае вы можете потерять зрение навсегда.
        \subsection{Что должно быть в автомобильной аптечке.}
        С 1 января 2021 г. устанавливается такая комплектация транспортного средства:
        \begin{itemize}
            \item Нестерильный марлевый бинт
            \item Рулонный пластырь
            \item Ножницы
            \item Одноразовые медицинские маски
            \item Одноразовые медицинские перчатки из непромокаемого материала (размер М или больше) -- 2 пары.
            \item Стерильные марлевые салфетки размером 16х14 см в герметичной упаковке -- 2 пачки.
            \item Кровоостанавливающий эластичный жгут 1 шт.
            \item Маска для безопасного исскуственного дыхания -- 1 шт.
            \item Инструкция по использованию предметов и оказанию первой помощи -- 1 шт. 
        \end{itemize}

















        \section{Защита наслеения и территорий при авариях на радиационно (ядерно) опасных объектах с выбросом радиоактивных веществ в ОКС.}
        \subsection{Аварии на радиационно (ядерно) опасных объектах и радиоактивное заражение ОКС.}

        К радиационно опасным объектам (РОО) относятся объекты, на которых хранятся, перерабатываются, используются или транспортируются радиоактивные вещества, при аварии на которых может произойти облучение ионизирующими излучениями людей, с/х животных и радиоактиыное загрязнение ОКС.

        В состав РОО по ряду критериев входят и так называемые ядерно опасные объекты (ЯОО), представляющие наибольшую опасность при авариях.

        Под ЯОО понимаются объекты, имеющие значительное количество ядерноделящихся матриалов (ЯМ) в различных физических состояниях и формах, потенциальная опасность функционирования которых заключается в возможности возникновения аварийных ситуациях самоподдерживающейся цепной ядерной реакциеи (СИЯР).
        \begin{itemize}
            \item Объекты ядерного топливного цикла (АС) и ядерные энергетические установки различного назначения.
            \item Научно-исследовательские реакторы
            \item Объекты ядерно-оружейного комплекса
        \end{itemize} 
        \subsection{Классификация РОО (ЯОО)}
        По типу:
        \begin{itemize}
            \item Атомные станции \begin{enumerate}
                \item Атомные электростанции (АЭС)
                \item Атомные теплоэлектростанции (АТЭЦ)
                \item Атомные станции теплоснабжения (АЭТС)
                \item Мобильные (плачучие) АЭС
                \end{enumerate}
            \item Предприятия ядерного топливного цикла 
                \begin{enumerate}
                    \item Уранодобывающие предприятия
                    \item Заводы по переработке и обогащению ядерного топлива
                    \item Хранилища радиоактивных отходов, склады ядерных материалов.
                \end{enumerate}
            \item Ядерные реакторы (корабельные, р/космические, исследовательские, медицинские)
            \item Объекты ядерного оружейного комплекса
            \item Транспортные средства с рад. груз. радиац. источники
        \end{itemize}
        По потенциальной радиоактивной опасности:
        \begin{itemize}
            \item К 1 категории относятся радиационные объекты, при аварии на которых возможно их радиационное воздействие на население и могут потребоваться меры по его защите.
            \item Ко 2 категории объектов радиационное воздействиепри аварии ограничивается территорией санитарно-защитной зоны.
            \item К 3 категории относятся объекты, радиационное воздействие при аварии которых ограничивается территорией объекта.
            \item К 4 категории относятся объекты, радиационное воздействие от которых при аварии ограничивается помещениями, где проводятся работы с источниками излучения.
        \end{itemize}
        \subsection{Поражающте факторы аварии}
        На объекте: 
        \begin{itemize}
            \item Ионизирующее излучение как непосредственно при выбросе радиоактивных веществ, так и при радиоактивном загрязнении территории объекта.
            \item тепловое воздействие (при наличии пожаров или аварии)
            \item Ударная волна (при наличии взрыва или аварии)
        \end{itemize}
        Вне объекта:
        \begin{itemize}
            \item Ионизирующее излучение как поражающий фактор радиационного загрязнения ОКС.
        \end{itemize}
        Из всех поражающих факторов, возникающих в результате аварии на РОО (ЯОО) наибольшую и специфичную опасность для жизни и здоровья людей представляет ионизирующее излучение (ИИ).
        \subsection{Радиация и радиактивность.}
        В самом широком смысле слова, радиация (лат. "сияние", "излучение") -- это процесс распространения энергии в пространстве в форме различных волн и частиц. Сюда можно отнести инфракрасное (тепловое) излучение, ультрафиолетовое, видимое световое, а также различные типы ионизирующего излучения. Наибольший интерес с точик зрения здоровья и безопасности жизенедеятельности представляет ионизирующая радиация, т.е. виды излучений, способные вызывать ионизация вещества, на которое они воздействуют.

        Источники ионизирующего излучения: 
        \begin{itemize}
            \item Единственная радиактивность
            \item Исскуственная радиактивность
            \item Космическое излучение и солнечная радиация
            \item Объекты атомной энергетики
            \item Излучение земной коры
            \item Медицинские процедуры, связанные с применением р/изотопов
            \item Атомное оружие (испытания)
            \item Радон
            \item Приборы разведки полезных ископаемых, использование р/изотопов
        \end{itemize}
        \begin{center}
            \textbf{Общий фон радиации от естественных источников облучения.} 
        \end{center}

        Естественный радиационный фон - доза излучения, создаваемая космическими лучами и излучением природных радионуклидов, естественно распределенных в земле, воде, воздухе, других элементах биосферы, пищевых продуктах и организме человека. Радиоактивный фон присутствует везде и всегда - где-то его уровень больше обычной нормы, где-то меньше. 

        Естественный радиационный фон везде свой, в зависимости от высоты территории над уровнем моря и геологического строения каждого конкретного района. Безопасным считается уровень радиации до величины, приблизительно 0.5 микрозиверт в ч а с (до 50 микрорентген в час).
        
        Всего существует 3 источника естественной радиации:
        \begin{itemize}
            \item Космическое излучение и солнечная радиация
            \item Излучение земной коры
            \item Излучение земной коры.
        \end{itemize}
        \begin{center}
            \textbf{Источники исскуственной радиации} 
        \end{center}
        В отличие от естественных источников радиации, искусственная радиоактивность возникла и распространяется исключительно силами людей. К основным техногенным радиоактивным источникам относят ядерное оружие, промышленные отходы, АЭС,  медицинское оборудование, предметы старины, вывезенные из «запретных» зон после аварии Чернобыльской АЭС, некоторые драгоценные камни. 
        \begin{center}
            \textbf{Виды облучений}
            
            В зависимости от месторасположения источника излучения
        \end{center}
        \textbf{Внешнее облучение.}
        От наружных источников излучения (космические лучи, воздействие природных или искусственных излучателей). Внутреннее — от радиоактивных веществ, попадающих внутрь организма человека с вдыхаемым воздухом, продуктами питания, с водой.\\
        \textbf{Внутреннее облучение.}
        Если источники радиации при внешнем облучении находится вне организма, то в данной ситуации они располагаются непосредственно внутри человека. Способы проникновения туда могут быть различными: через дыхательную систему, по пищеварительному тракту, в связи с повреждением целостности кожных покровов и др.

        Такой риск появляется в ситуации, если человек находиться в обстановке, где присутствует открытый источник излучения или окружающая среда загрязнена.
        \begin{center}
            \textbf{Дозы внешнего и внутреннего облучения} 
        \end{center}
        0,5 мЗв от внешнего гамма-излучения (от 0,3 до 0,6 мЗв, в зависимости от радионуклидного состава окружения — почвы, стройматериалов и т. п.); 1,2 мЗв внутреннего облучения от ингалируемых атмосферных радионуклидов, главным образом радона (от 0,2 до 10 мЗв, в зависимости от местной концентрации радона в воздухе);

        Уммарная доза внешнего эффективного облучения -- 0.7-1 мЭв в год.

        Интересный факт: эффективная доза от внутреннего облучения за счет естественных источников в среднем примерно в два раза превышает дозу внешнего облучения от них.

        Следовательно, суммарная доза внешнего и внутреннего облучения от естественных источников радиации в среднем равна 2-3 мЭв в год.
        \subsection{Санитарная обработка}
        \begin{center}
            \textbf{Частичная санитарная обработка.} 
        \end{center}

        Частичная санитарная обработка населения при радиоактивном заражении местности  (при химическом заражении      аэровзвесями и каплями АХОВ) проводится в ближайшем укрытии и заключается в том, чтобы, не снимая противогаза провести обработку открытых участков тела, загрязненных участков одежды, обуви и лицевой части маски противогаза. Обработка проводится раствором индивидуального противохимического пакета, а при его отсутствии — подручными средствами (водой, ветошью).
        \begin{center}
            \textbf{Полная санитарная обработка.} 
        \end{center}
        
        Полная санитарная обработка, так же как и частичная, заключается в удалении радиоактивных и отравляющих веществ или бактериальных средств, но в отличие от неё носит характер заключительной меры профилактики поражения людей и сохранения их работоспособности. Её выполняют более тщательно, при этом обрабатывают не только отдельные заражённые участки кожи, но и всю поверхность тела водой с моющими средствами. Особенно тщательно промыть (прополоскать) глаза, рот, горло.  После проверки дозиметрического контроля оденьте чистое белье, одежду и обувь. 

        Проводят ПСО на границе «Грязной» и «Чистой» территории.
        Люди, пришедшие в заражённой одежде и нуждающиеся в полной санитарной обработке, направляются в раздевалки, где снимают и передают свою одежду в специально оборудованное помещение для сбора загрязнённой одежды и подготовки её к обеззараживанию. Далее все прибывшие проходят в помещение, где медицинский персонал осматривает поражённых, помогает им в обработке слизистых оболочек глаз, носа и рта, а также оказывает нуждающимся необходимую медицинскую помощь.

        \subsection{Порядок действий при получении оповещения о радиационном заражении района}
        Получив информацию о РЗ района: 
        \begin{enumerate}
            \item Защитить органы дыхания простейшим средством (платок, шарф).
            \item Вернуться домой.
            \item Оповестить соседей по площадке о ЧС и порядке действий.
            \item Снять верхнею одежду и обувь, упаковать их в пакет, войти в квартиру.
            \item Закрыть  окна (откл. вентиляцию),вкл. радио, принять таблетку йодистого калия   и  душ.
            \item Заменить платок ВМП  (респиратором)
            \item Провести дополнительную герметизацию помещения.
            \item Имеющиеся продукты поместить в полиэтиленовые пакеты и поместить в холодильник (шкаф). Сделать запас питьевой воды (2-3 л).
            \item Проводите только влажную уборку помещения
            \item Подготовиться к эвакуации: (документы, денежные средства, два-три комплекта н/белья, лекарства,телефон(зарядку) продукты, воду, простейшие средства индивидуальной  защиты кожи, туалетные принадлежности  и тд.) Уточнить места сбора для последующей эвакуации , маршрут и район сосредоточения. Сообщить родным, близким и соседям ( при необходимости оказать помощь пожилым)
            Не более 5 мзв в год.
        \end{enumerate}

        С поучением сигнала на Эвакуацию: продублировать информацию соседям, перекрыть водоснабжение и газ,  очистить холодильник от продуктов и вынести мусор, еще раз проверить по списку комплект вещей (документов), продуктов и оценить ваши возможности по «грузоподъёмности» багажа и его доставки на сборно эвакуационный пункт. 

        На сборно эвакуационном пункте:

        Перед посадкой при необходимости с отдельными гражданами может проводиться частичная санитарная обработка.

        При посадке в транспорт (транспорт «грязный») средства защиты не снимать (вы находитесь в зараженном районе). 

        На границе «чистого района» эвакуируемые выходят из машин и направляются в пункты полной санитарной обработки.

        Люди, пришедшие в заражённой одежде и нуждающиеся в полной санитарной обработке, направляются в раздевалки,
        где снимают и передают свою одежду в специально оборудованное помещение для сбора загрязнённой одежды и подготовки её к обеззараживанию. Далее все прибывшие проходят в помещение, где медицинский персонал осматривает поражённых, помогает им в обработке слизистых оболочек глаз, носа и рта, а также оказывает нуждающимся необходимую медицинскую помощь и выходят на ЧИСТУЮ территорию.

        ПСО  носит характер заключительной меры профилактики поражения людей и сохранения их работоспособности. Её выполняют более тщательно, при этом обрабатывают не только отдельные заражённые участки кожи, но и всю поверхность тела водой с моющими средствами. Особенно тщательно промыть (прополоскать 2\% раствором пищевой соды) глаза, рот, горло.  После проверки дозиметрического контроля оденьте чистое белье, одежду и обувь.

        При заражении радиоактивными веществами её выполняют в следующем порядке: одежду вытряхивают, обметывают, выколачивают; обувь протирают влажной ветошью; открытые участки шеи, рук обмывают; лицевую часть противогаза протирают и только после этого снимают. Если были надеты респиратор, ПТМ, ватно-марлевая повязка – тоже снимают. Затем моют лицо, полощут рот и горло.

        Когда воды недостаточно, можно открытые участки тела и лицевую часть противогаза протереть влажным тампоном, причем только в одном направлении, все время поворачивая его. Зимой для этих целей можно использовать незараженный снег. 

        Частичная санитарная обработка не обеспечивает полного обеззараживания и тем самым не гарантирует людям полную защиту от поражения радиоактивными, отравляющими, сильнодействующими ядовитыми веществами и бактериальными средствами. 

        Поэтому при первой возможности производят полную санитарную обработку. При полной санитарной обработке всё тело обмывается тёплой водой с мылом и мочалкой, обязательно меняется бельё и одежда. Проводится на стационарных обмывочных пунктах, в банях, душевых павильонах или специально развёртываемых обмывочных площадках и пунктах специальной.

        Летом полную санитарную обработку можно осуществить в незараженных проточных водоёмах. Все обмывочные пункты и площадки, как правило, имеют три отделения: раздевальное, обмывочное и одевальное. Кроме того, при обмывочном пункте может быть отделение обеззараживания одежды. Лица, прибывшие на санитарную обработку, входом в раздевальное отделение снимают верхнюю одежду и средства защиты (кроме противогаза) и складывают их в указанное место. Здесь же снимают бельё, проходят медицинский осмотр, дозиметрический контроль, тем, у кого подозревают инфекционные заболевания, измеряют температуру. Одежду, зараженную РВ выше допустимых норм, а также СДЯВ, ОВ и бактериальными средствами, складывают в резиновые мешки и отправляют на станцию обеззараживания одежды. Перед входом в обмывочное отделение пораженные снимают противогазы и обрабатывают слизистые оболочки 2\% раствором питьевой соды. Каждому выдается 25-40 г мыла и мочалка. Особенно тщательно требуется вымыть голову, шею, руки. Под каждой душевой сеткой одновременно моются 2 человека. Температура воды 38-40°С. При заражении бактериальными средствами перед входом в раздевальное отделение одежду подвергают орошению 0,5\% раствором монохлорамина, а руки и шею обрабатывают 2\% раствором. Затем, получив мочалку и мыло, снимают противогаз и переходят в обмывочное отделение. После выхода из него производится вторичный медицинский осмотр и дозиметрический контроль. Если радиоактивное заражение всё ещё выше допустимых норм, людей возвращают на повторную обработку. В одевальном отделении все получают свою обеззараженную одежду или из запасного фонда и одеваются. 

        Продолжительность санобработки в пределах 30 минут (раздевание – 5 минут, мытьё под душем – 15 минут, одевание – 10 минут). Для увеличения пропускной способности душевой очередная смена людей раздевается ещё до окончания мытья предыдущей и занимает место под душем по мере их освобождения.
        
        Если благоустроенные санитарно-обмывочные пункты отсутствуют, то полную санитарную обработку проводят в банях, душевых павильонах, дооборудованных таким образом, чтобы поток людей двигался только в одном направлении, и не происходило пересечение.

        \section{Экология}
        \subsection{Экология. Транспорт}

        Урбанизация - одна из важнейших демографических тенденций нашего времени. С ростом числа и размеров городов стремительно нарастают экологические проблемы, которые определяют жизнь среднестатистического жителя города.  

        Анализ данных литературы показывает, что в основе экологических проблем мегаполисов лежат несколько объективных причин:

        во-первых, высокая концентрация населения на весьма ограниченной территории; 

        во-вторых, население, чтобы обеспечить себя материально должно работать, что предполагает концентрацию огромного промышленного потенциала на\\ определенной территории; 

        и, в-третьих, мегаполис должен иметь мощную автотранспортную\\ индустрию, без чего не может быть обеспечена нормальная жизнь города.

        Концентрация промышленного потенциала и автотранспорта неминуемо\\ приводит к загрязнению городской среды и ухудшению условий жизнедеятельности и безопасности здоровья горожан. Достаточно сказать, что в Москве на каждого жителя приходится по 46 кг вредных веществ в год, а в Казани в 2019 г. этот показатель составил в 2012г. 52,1 кг/год. 
 
        Несмотря на то что все крупные города России отличаются друг от друга по своему экологическому статусу, тем не менее, основные проблемы экологии крупных городов, связаны с чрезмерной концентрацией на сравнительно небольших территориях населения, транспорта и промышленных предприятий, с образованием антропогенных ландшафтов, очень далеких от состояния экологического равновесия.

        В настоящее время автотранспорт является одним из основных источников  загрязнения  атмосферного воздуха. Над крупными городами атмосфера содержит в 10 раз больше аэрозолей и в 25 раз больше газов. При этом 60-70\% газового загрязнения дает автомобильный транспорт.

        Сегодня мировой автомобильный парк превышает 600 млн. единиц, из которых 83-85\% составляют легковые, 15-17\% - грузовые автомобили и автобусы. Если их поставить бампер к бамперу, то получилась бы лента длиной 4 млн. км, которой можно было бы 100 раз опоясать земной шар по экватору.
 
        Доля транспортных средств в загрязнении воздуха в городах достигает\\ 70-90\%, что создает достаточно устойчивые и обширные зоны, внутри которых санитарно-гигиенические нормативы загрязнения воздуха превышены в несколько раз.

        Состав выхлопных газов автотранспорта зависит от типа двигателя, режима работы, технического состояния и качества топлива. В настоящее время изучено более 200 компонентов, входящих в состав отработанных газов автотранспорта. 

        По объему наибольший удельный вес имеют оксид углерода (0,5-10\%), оксиды азота (до 0,8\%), несгоревшие углеводороды (0,2-3,0\%), альдегиды (до 0,2\%) и сажа. Токсичность отработавших газов карбюраторных двигателей обуславливается главным образом содержанием окиси углерода и оксидов азота, а дизельных двигателей - оксидов азота и сажи.

        В среднем, автомобиль потребляет в год 2 т бензина и выбрасывает в воздух  20-25 тыс. м3 продуктов сгорания, в которых содержится 700 кг СО, 40 кг NО, 230 кг углеводородов и 2-5 кг твердых частиц.

        Уровень загрязнения атмосферного воздуха отработанными газами автотранспорта зависит также от режима его работы, от скорости движения транспорта, от интенсивности движения автомобилей, ширины и рельефа улицы, скорости ветра, доли грузового транспорта и автобусов в общем потоке и других факторов.

        Доля выбросов автотранспорта (\%) от общего количества выбрасываемых веществ.

        Окись углерода:

        Санкт-Петербург $\tab[1cm]$  88\%

        Мадрид $\tab[1cm]$ 95\%

        Токио $\tab[1cm]$ 99\%






        \section{Пожары}
        \subsection{
        Федеральный закон от 21 декабря 1994 г. N 69-ФЗ "О пожарной безопасности" (с изменениями и дополнениями от 2021гг.)}

        Статья 1. Основные понятия

        2. При обнаружении пожара или признаков горения в здании, помещении, на территории (задымление, запах гари, повышение температуры воздуха и др.) должностным лицам, индивидуальным предпринимателям, гражданам Российской Федерации, иностранным гражданам, лицам без гражданства (далее - физические лица) необходимо:

        (в ред. Постановления Правительства РФ от 24.10.2022 N 1885)
        немедленно сообщить об этом по телефону в пожарную охрану с указанием наименования объекта защиты, адреса места его расположения, места возникновения пожара, а также фамилии сообщающего информацию;
        принять меры по эвакуации людей, а при условии отсутствия угрозы жизни и здоровью людей меры по тушению пожара в начальной стадии.

        В целях настоящего Федерального закона применяются следующие понятия:

        пожарная безопасность - состояние защищенности личности, имущества, общества и государства от пожаров;
        пожар - неконтролируемое горение, причиняющее материальный ущерб, вред жизни и здоровью граждан, интересам общества и государства;
        
        обязательные требования пожарной безопасности (далее - требования пожарной безопасности) - специальные условия социального и (или) технического характера, установленные в целях обеспечения пожарной безопасности федеральными законами и иными нормативными правовыми актами Российской Федерации, а также нормативными документами по пожарной безопасности;

        нарушение требований пожарной безопасности - невыполнение или ненадлежащее выполнение требований пожарной безопасности;

        противопожарный режим - совокупность установленных нормативными правовыми актами Российской Федерации, нормативными правовыми актами субъектов Российской Федерации и муниципальными правовыми актами по пожарной безопасности требований пожарной безопасности, определяющих правила поведения людей, порядок организации производства и (или) содержания территорий, земельных участков, зданий, сооружений, помещений организаций и других объектов защиты в целях обеспечения пожарной безопасности;

        меры пожарной безопасности - действия по обеспечению пожарной безопасности, в том числе по выполнению требований пожарной безопасности;

        пожарная охрана - совокупность созданных в установленном порядке органов управления, подразделений и организаций, предназначенных для организации профилактики пожаров, их тушения и проведения возложенных на них аварийно-спасательных работ;

        ведомственный пожарный контроль - деятельность ведомственной пожарной охраны по проверке соблюдения организациями, подведомственными соответствующим федеральным органам исполнительной власти, требований пожарной безопасности и принятие мер по результатам проверки;


        подтверждение соответствия в области пожарной безопасности - документальное удостоверение соответствия продукции или иных объектов, выполнения работ и оказания услуг требованиям технических регламентов, документов по стандартизации, принятых в соответствии с законодательством Российской Федерации о стандартизации, норм пожарной безопасности или условиям договоров;


        нормативные документы по пожарной безопасности - национальные стандарты Российской Федерации, своды правил, содержащие требования пожарной безопасности, а также иные документы, содержащие требования пожарной безопасности;


        профилактика пожаров - совокупность превентивных мер, направленных на исключение возможности возникновения пожаров и ограничение их последствий;


        первичные меры пожарной безопасности - реализация принятых в установленном порядке норм и правил по предотвращению пожаров, спасению людей и имущества от пожаров;

        пожарно-спасательный гарнизон - совокупность расположенных на определенной территории органов управления, подразделений и организаций независимо от их ведомственной принадлежности и форм собственности, к функциям которых отнесены профилактика и тушение пожаров, а также проведение аварийно-спасательных работ;

        организация тушения пожаров - совокупность оперативно-тактических и\\ инженерно-технических мероприятий (за исключением мероприятий по обеспечению первичных мер пожарной безопасности), направленных на спасение людей и имущества от опасных факторов пожара, ликвидацию пожаров и проведение аварийно-спасательных работ;

        особый противопожарный режим - дополнительные требования пожарной безопасности, устанавливаемые органами государственной власти или органами местного самоуправления в случае повышения пожарной опасности на соответствующих территориях;

        локализация пожара - действия, направленные на предотвращение возможности дальнейшего распространения горения и создание условий для его ликвидации имеющимися силами и средствами;

        координация в области пожарной безопасности - деятельность по обеспечению взаимосвязи (взаимодействия) и слаженности элементов системы обеспечения пожарной безопасности;

        противопожарная пропаганда - информирование общества о путях обеспечения пожарной безопасности;

        обучение мерам пожарной безопасности - организованный процесс по формированию знаний, умений, навыков граждан в области обеспечения пожарной безопасности в системе общего, профессионального и дополнительного образования, в процессе трудовой и служебной деятельности, а также в повседневной жизни;

        управление в области пожарной безопасности - деятельность органов, участвующих в соответствии с законодательством Российской Федерации в обеспечении пожарной безопасности;

        зона пожара - территория, на которой существует угроза причинения вреда жизни и здоровью граждан, имуществу физических и юридических лиц в результате воздействия опасных факторов пожара и (или) осуществляются действия по тушению пожара и проведению аварийно-спасательных работ, связанных с тушением пожара;

        независимая оценка пожарного риска (аудит пожарной безопасности) - оценка соответствия объекта защиты требованиям пожарной безопасности и проверка соблюдения организациями и гражданами противопожарного режима, проводимые не заинтересованным в результатах оценки или проверки экспертом в области оценки пожарного риска;

        эксперт в области оценки пожарного риска - должностное лицо, аттестованное в порядке, установленном Правительством Российской Федерации, осуществляющее деятельность в области оценки пожарного риска, обладающее специальными знаниями в области пожарной безопасности, необходимыми для проведения независимой оценки пожарного риска (аудита пожарной безопасности), и уполномоченное на подписание заключения о независимой оценке пожарного риска (аудите пожарной безопасности);

        заведомо ложное заключение о независимой оценке пожарного риска (аудите пожарной безопасности) - заключение о независимой оценке пожарного риска (аудите пожарной безопасности), подготовленное без проведения независимой оценки пожарного риска (аудита пожарной безопасности) или подготовленное после ее проведения, но противоречащее содержанию материалов, представленных эксперту в области оценки пожарного риска, состоянию пожарной безопасности объекта защиты, в отношении которого проведена независимая оценка пожарного риска (аудит пожарной безопасности), фактическому соблюдению организациями и гражданами противопожарного режима;

        ландшафтный (природный) пожар - неконтролируемый процесс горения,\\ стихийно возникающий и распространяющийся в природной среде, охватывающий различные компоненты природного ландшафта;

        лесной пожар - разновидность ландшафтного (природного) пожара,\\ распространяющегося по лесу.

        Статья 3. Система обеспечения пожарной безопасности
        Система обеспечения пожарной безопасности - совокупность сил и средств, а также мер правового, организационного, экономического, социального и научно-технического характера, направленных на профилактику пожаров, их тушение и проведение аварийно-спасательных работ.
        Основными элементами системы обеспечения пожарной безопасности являются органы государственной власти, органы местного самоуправления, организации, граждане, принимающие участие в обеспечении пожарной безопасности в соответствии с законодательством Российской Федерации.
        Основные функции системы обеспечения пожарной безопасности: нормативное правовое регулирование и осуществление государственных мер в области пожарной безопасности;
        создание пожарной охраны и организация ее деятельности;
        разработка и осуществление мер пожарной безопасности;
        реализация прав, обязанностей и ответственности в области пожарной безопасности;
        проведение противопожарной пропаганды и обучение населения мерам пожарной безопасности;
        содействие деятельности добровольных пожарных, привлечение населения к обеспечению пожарной безопасности; 
        научно-техническое обеспечение пожарной безопасности;
        информационное обеспечение в области пожарной безопасности;

        осуществление федерального государственного пожарного надзора и других контрольных функций по обеспечению \\пожарной безопасности;
        производство пожарно-технической продукции;
        осуществление деятельности в области пожарной безопасности;
        лицензирование отдельных видов деятельности и подтверждение соответствия продукции и услуг в области пожарной безопасности (далее - подтверждение соответствия);
        тушение пожаров и проведение аварийно-спасательных работ;
        учет пожаров и их последствий;
        установление особого противопожарного режима;
        организация и осуществление профилактики пожаров.
        Статья 4. Виды и основные задачи пожарной охраны
        Пожарная охрана подразделяется на следующие виды:
        государственная противопожарная служба;
        муниципальная пожарная охрана;
        ведомственная пожарная охрана;
        частная пожарная охрана;
        добровольная пожарная охрана.
        Основными задачами пожарной охраны являются:
        организация и осуществление профилактики пожаров;
        спасение людей и имущества при пожарах, оказание первой помощи;
        организация и осуществление тушения пожаров и проведения аварийно-спасательных работ.
        К действиям по предупреждению, ликвидации социально-политических, межнациональных конфликтов и массовых беспорядков пожарная охрана не привлекается.
        Организация управления в области пожарной безопасности и координация деятельности пожарной охраны осуществляются федеральным органом исполнительной власти, уполномоченным на решение задач в области пожарной безопасности.
        Статья 11.1. Муниципальная пожарная охрана

        Муниципальная пожарная охрана создается органами местного самоуправления на территории муниципальных образований.

        Цель, задачи, порядок создания и организации деятельности муниципальной пожарной охраны, порядок ее взаимоотношений с другими видами пожарной охраны определяются органами местного самоуправления.
        Статья 19. Полномочия органов местного самоуправления в области пожарной безопасности
        К полномочиям органов местного самоуправления поселений, муниципальных, городских округов, внутригородских районов по обеспечению первичных мер пожарной безопасности в границах сельских населенных пунктов относятся:
        создание условий для организации добровольной пожарной охраны, а также для участия граждан в обеспечении первичных мер пожарной безопасности в иных формах;
        создание в целях пожаротушения условий для забора в любое время года воды из источников наружного водоснабжения, расположенных в сельских населенных пунктах и на прилегающих к ним территориях;
        оснащение территорий общего пользования первичными средствами тушения пожаров и противопожарным инвентарем;
        организация и принятие мер по оповещению населения и подразделений Государственной противопожарной службы о пожаре;
        принятие мер по локализации пожара и спасению людей и имущества до прибытия подразделений Государственной противопожарной службы;
        включение мероприятий по обеспечению пожарной безопасности в планы, схемы и программы развития территорий поселений, муниципальных и городских округов;
        оказание содействия органам государственной власти субъектов Российской Федерации в информировании населения о мерах пожарной безопасности, в том числе посредством организации и проведения собраний населения;
        установление особого противопожарного режима в случае повышения пожарной опасности.
        К полномочиям органов местного самоуправления поселений, муниципальных, городских округов, внутригородских районов по обеспечению первичных мер пожарной безопасности в границах городских населенных пунктов относятся:
        создание условий для организации добровольной пожарной охраны, а также для участия граждан в обеспечении первичных мер пожарной безопасности в иных формах;
        включение мероприятий по обеспечению пожарной безопасности в планы, схемы и программы развития территорий поселений, муниципальных и городских округов;
        оказание содействия органам государственной власти субъектов Российской Федерации в информировании населения о мерах пожарной безопасности, в том числе посредством организации и проведения собраний населения;
        установление особого противопожарного режима в случае повышения пожарной опасности.
        К полномочиям органов местного самоуправления муниципальных районов относится обеспечение первичных мер пожарной безопасности в границах муниципальных районов за границами городских и сельских населенных пунктов.
        Вопросы организационно-правового, финансового, материально-технического обеспечения первичных мер пожарной безопасности поселений, муниципальных районов, муниципальных, городских округов, внутригородских районов устанавливаются нормативными актами органов местного самоуправления.
        В субъектах Российской Федерации - городах федерального значения полномочия органов местного самоуправления, предусмотренные настоящим Федеральным законом, в соответствии с законами указанных субъектов Российской Федерации осуществляются органами государственной власти субъектов Российской Федерации - городов федерального значения.
        См. комментарии к статье 19 настоящего Федерального закона
        Глава IV. Обеспечение пожарной безопасности (ст.ст. 20 - 33)
        Глава IV. Обеспечение пожарной безопасности
        Статья 21. Разработка и реализация мер пожарной безопасности

        Меры пожарной безопасности разрабатываются в соответствии с законодательством Российской Федерации по пожарной безопасности, а также на основе опыта борьбы с пожарами, оценки пожарной опасности веществ, материалов, технологических процессов, изделий, конструкций, зданий и сооружений.

        Изготовители (поставщики) веществ, материалов, изделий и оборудования в обязательном порядке указывают в соответствующей технической документации показатели пожарной опасности этих веществ, материалов, изделий и оборудования, а также меры пожарной безопасности при обращении с ними.

        Разработка и реализация мер пожарной безопасности для организаций, зданий, сооружений и других объектов, в том числе при их проектировании, должны в обязательном порядке предусматривать решения, обеспечивающие эвакуацию людей при пожарах.

        Для производств в обязательном порядке разрабатываются планы тушения пожаров, предусматривающие решения по обеспечению безопасности людей.

        Меры пожарной безопасности для населенных пунктов и территорий административных образований разрабатываются и реализуются соответствующими органами государственной власти, органами местного самоуправления.
        Статья 22. Тушение пожаров и проведение аварийно-спасательных работ

        Тушение пожаров представляет собой действия, направленные на спасение людей, имущества и ликвидацию пожаров.

        Проведение аварийно-спасательных работ, осуществляемых пожарной охраной, представляет собой действия по спасению людей, имущества и (или) доведению до минимально возможного уровня воздействия взрывоопасных предметов, опасных факторов, характерных для аварий, катастроф и иных чрезвычайных ситуаций.
        При тушении пожаров с участием других видов пожарной охраны функции по координации деятельности других видов пожарной охраны возлагаются на федеральную противопожарную службу.

        Порядок привлечения сил и средств подразделений \\пожарной охраны, пожарно-спасательных гарнизонов для тушения пожаров и проведения\\ аварийно-спасательных работ утверждается федеральным органом исполнительной власти, уполномоченным на решение задач в области пожарной безопасности.
        Выезд подразделений пожарной охраны на тушение пожаров и проведение аварийно-спасательных работ в населенных пунктах и организациях осуществляется в безусловном порядке.

        Тушение пожаров и проведение аварийно-спасательных работ осуществляются на безвозмездной основе, если иное не установлено законодательством Российской Федерации.

        Для приема сообщений о пожарах и чрезвычайных ситуациях используются единый номер вызова экстренных оперативных служб "112" и телефонный номер приема сообщений о пожарах и чрезвычайных ситуациях, назначаемый федеральным органом исполнительной власти в области связи.

        При тушении пожаров и проведении аварийно-спасательных работ силами подразделений пожарной охраны, привлеченными силами и средствами единой государственной системы предупреждения и ликвидации чрезвычайных ситуаций проводятся необходимые действия для обеспечения безопасности людей, спасения имущества, в том числе:

        проникновение в места распространения (возможного распространения)\\ опасных факторов пожаров, а также опасных проявлений аварий, катастроф и иных чрезвычайных ситуаций;

        создание условий, препятствующих развитию пожаров, а также аварий, катастроф и иных чрезвычайных ситуаций и обеспечивающих их ликвидацию;

        использование при необходимости дополнительно имеющихся в наличии у собственника средств связи, транспорта, оборудования, средств пожаротушения и огнетушащих веществ с последующим урегулированием вопросов, связанных с их использованием, в установленном порядке;

        ограничение или запрещение доступа к местам пожаров, а также зонам аварий, катастроф и иных чрезвычайных ситуаций, ограничение или запрещение движения транспорта и пешеходов на прилегающих к ним территориях;

        охрана мест тушения пожаров, а также зон аварий, катастроф и иных чрезвычайных ситуаций (в том числе на время расследования обстоятельств и причин их возникновения);

        эвакуация с мест пожаров, аварий, катастроф и иных чрезвычайных ситуаций людей и имущества, оказание первой помощи;
        приостановление деятельности организаций, оказавшихся в зонах воздействия опасных факторов пожаров, опасных проявлений аварий, если существует угроза причинения вреда жизни и здоровью работников данных организаций и иных граждан, находящихся на их территориях.
        Непосредственное руководство тушением пожара осуществляется руководителем тушения пожара - прибывшим на пожар старшим оперативным должностным лицом пожарной охраны (если не установлено иное), которое управляет на принципах единоначалия личным составом пожарной охраны, участвующим в тушении пожара, а также привлеченными к тушению пожара силами.
        Руководитель тушения пожара отвечает за выполнение задачи, за безопасность личного состава пожарной охраны, участвующего в тушении пожара, и привлеченных к тушению пожара сил.
        Руководитель тушения пожара определяет зону пожара, устанавливает границы территории, на которой осуществляются действия по тушению пожара и проведению аварийно-спасательных работ, порядок и особенности осуществления указанных действий, принимает решение о спасении людей и имущества, привлечении при необходимости к тушению пожара дополнительных сил и средств, в том числе единой государственной системы предупреждения и ликвидации чрезвычайных ситуаций, устанавливает порядок управления действиями подразделений пожарной охраны на месте пожара и привлеченных к тушению пожара сил, производит расстановку прибывающих сил и средств на месте пожара, организовывает связь в зоне пожара с участниками тушения пожара и привлеченными к тушению пожара и проведению аварийно-спасательных работ силами, принимает меры по сохранению вещественных доказательств, имущества и вещной обстановки на месте пожара для последующего установления причины пожара. При необходимости руководитель тушения пожара принимает иные решения, в том числе ограничивающие права должностных лиц и граждан на указанной территории.

        Указания руководителя тушения пожара обязательны для исполнения всеми должностными лицами и гражданами на территории, на которой осуществляются действия по тушению пожара.
        Никто не вправе вмешиваться в действия руководителя тушения пожара или отменять его распоряжения при тушении пожара.
        Личный состав пожарной охраны, иные участники тушения пожара, ликвидации аварии, катастрофы, иной чрезвычайной ситуации, действовавшие в условиях крайней необходимости и (или) обоснованного риска, от возмещения причиненного ущерба освобождаются.
        При тушении пожара личный состав пожарной охраны должен принимать меры по сохранению вещественных доказательств и имущества.
        Статья 25. Противопожарная пропаганда и обучение мерам пожарной безопасности

        Противопожарная пропаганда осуществляется через средства массовой информации, посредством издания и распространения специальной литературы и рекламной продукции, проведения тематических выставок, смотров, конференций и использования других не запрещенных законодательством Российской Федерации форм информирования населения. Противопожарную пропаганду проводят органы государственной власти, федеральный орган исполнительной власти, уполномоченный на решение задач в области пожарной безопасности, органы местного самоуправления и организации.

        Порядок, виды, сроки обучения лиц, осуществляющих трудовую или служебную деятельность в организациях по программам противопожарного инструктажа, а также требования к содержанию указанных программ, порядок их утверждения и согласования определяются федеральным органом исполнительной власти, уполномоченным на решение задач в области пожарной безопасности.

        Дополнительное профессиональное образование в области пожарной\\ безопасности осуществляется в соответствии с законодательством Российской Федерации об образовании. Категории лиц, проходящих обучение по дополнительным профессиональным программам, определяются федеральным органом исполнительной власти, уполномоченным на решение задач в области пожарной безопасности.

        В образовательных организациях проводится обязательное обучение обучающихся мерам пожарной безопасности. Органами, осуществляющими управление в сфере образования, и пожарной охраной могут создаваться добровольные дружины юных пожарных. Порядок создания и деятельности добровольных дружин юных пожарных определяется федеральным органом исполнительной власти, осуществляющим функции по выработке и реализации государственной политики и нормативно-правовому регулированию в сфере общего образования, по согласованию с федеральным органом исполнительной власти, уполномоченным на решение задач в области пожарной безопасности.

        Статья 34. Права и обязанности граждан в области пожарной безопасности
        Граждане имеют право на:
        защиту их жизни, здоровья и имущества в случае пожара;
        возмещение ущерба, причиненного пожаром, в порядке, установленном действующим законодательством;
        участие в установлении причин пожара, нанесшего ущерб их здоровью и имуществу;
        получение информации по вопросам пожарной безопасности, в том числе в установленном порядке от органов управления и подразделений пожарной охраны;
        участие в обеспечении пожарной безопасности, в том числе в установленном порядке в деятельности добровольной пожарной охраны.


        Граждане обязаны:
        соблюдать требования пожарной безопасности;
        иметь в помещениях и строениях, находящихся в их собственности (пользовании), первичные средства тушения пожаров и противопожарный инвентарь в соответствии с правилами противопожарного режима и перечнями, утвержденными соответствующими органами местного самоуправления;
        при обнаружении пожаров немедленно уведомлять о них пожарную охрану;
        до прибытия пожарной охраны принимать посильные меры по спасению людей, имущества и тушению пожаров;
        оказывать содействие пожарной охране при тушении пожаров;
        выполнять предписания, постановления и иные законные требования должностных лиц государственного пожарного надзора;
        предоставлять в порядке, установленном законодательством Российской Федерации, возможность должностным лицам государственного пожарного надзора проводить обследования и проверки принадлежащих им производственных, хозяйственных и иных помещений и строений (за исключением жилых помещений), территорий, земельных участков в целях контроля за соблюдением требований пожарной безопасности и пресечения их нарушений.

        Правила противопожарного режима в Российской Федерации
        Постановление Правительства РФ от 16.09.2020 N 1479 (ред. от 30.03.2023) "Об утверждении Правил противопожарного режима в Российской Федерации"	

        \subsection{Порядок действий по единому сигналу.}
        Если сигнал застал вас на работе:

        1. Включить радио, радиотрансляционные и телевизионные приборы.

        2. Внимательно прослушать сообщение о сложившейся ситуации и порядке действий.
        Оцените обстановку!

        3. Действовать в соответствии с переданным сообщением.

        4. Доведите информацию до своих родных, близких, не забудьте о пожилых соседях. 

        Если сигнал застал вас на улице:
        
        1. Прослушать сообщение, передаваемое уличными громкоговорителями и подвижными средствами оповещения.

        2. Прочитать информационное сообщение на уличных светодиодных экранах, плазменных панелях, расположенных в местах массового пребывания людей. Оцените обстановку.

        3. Действовать в соответствии с переданным сообщением и принятым Решением!

        \subsection{Тема 6 
        «Защита населения и территорий при пожарах и взрывах на объектах инфраструктуры»}

        Пожар – неконтролируемое горение, причиняющее материальный ущерб, вред жизни и здоровью граждан, интересам общества и государства. (Федеральный закон "О пожарной безопасности" от 21.12.1994 N 69-ФЗ (в ред.01.01.2022г))

        Система обеспечения пожарной безопасности - совокупность сил и средств, а также мер правового, организационного, экономического, социального и \\научно-технического характера, направленных на профилактику пожаров, их тушение и проведение аварийно-спасательных работ.

        \begin{center}
            \textbf{Основные причины пожаров.} 
        \end{center}
        \begin{itemize}
            \item неосторожное обращение с огнем
            \item НПУиЭ печей
            \item  НПУиЭ электрооборудования
            \item Поджоги
            \item Шалость детей
            \item прочие причины
        \end{itemize}
        
        Объекты пожаров:
        \begin{itemize}
            \item транспортные средства
            \item Общественные здания
            \item Производственные здания
            \item Прочие объекты
        \end{itemize}
        
        \subsection{Опасные фактор пожара}

        Нагрев потоком выражается в ожогах открытых частей тела, легких и дыхательных путей.

        Ожоги 1-й степени повреждают только верхний слой кожи, называемый эпидермисом. Вызывает чувствительность, красноту, боль, лёгкую припухлость кожи. Волдырей нет.

        Ожоги 2-й степени повреждают эпидермис и слой, находящийся под ним, - дерму. Вызывает боль (часто сильную), волдыри, отёк. Поверхность обожжённого участка мокрая или сочащаяся. Кожа красная, с прожилками или пятнами.

        Ожоги 3-й и 4-й степени разрушают все слои кожи и могут повредить мышцы и кости. Вызывает потерю сознания, расстройство дыхания (лёгкие повреждаются из-за вдыхания дыма). Кожа побелевшая, покрасневшая, сероватая, потемневшая, обугленная. Боль слабая или отсутствует (при разрушении нервных окончаний). Мертвая кожа отслаивается. Возможен летальный исход.

        \subsection{Первыичные средства пожаротушения}

        \begin{definition}
            Первыичные средства пожаротушения -- это устройства, инструменты и материалы, предназначенные для локализации и (или) ликвидации загорания на начальной стадии (огнетушители, внутренний пожарный кран, вода, песок, кошма, асбестовое полотно, ведро, лопата и др.). 
        \end{definition}

        Тушение пожара — это работа пожарных-профессионалов, а борьба с загоранием посильна для неспециалистов.  Первичные средства применяются для борьбы с загоранием, но не с пожаром. Противостоять развившемуся пожару с их помощью невозможно и даже — опасно для жизни.

        \begin{center}
            \textbf{ОБЕСПЕЧЕНИЕ  ПЕРСОНАЛА ПВОО ОБЪЕКТА И НАСЕЛЕНИЯ СРЕДСТВАМИ ИНДИВИДУАЛЬНОЙ ЗАЩИТЫ }
        \end{center}

        Самоспасатель фильтрующий ГДЗК-EN:
        Срок службы в состоянии ожидания применения (гарантийный срок хранения) - 7 лет. T Защиты не менее 30 мин

        Самоспасатель ГДЗК «ГАРАНТ-1»: 
        гарантийный срок хранения) - 5 лет. T Защиты не менее 30 мин

        СПЕЦИАЛЬНАЯ ОГНЕСТОЙКАЯ НАКИДКА ШАНС

        \begin{center}
            \textbf{Огнетушители: типы, виды, классификация, область применения} 
        \end{center}

        Огнетушитель — это переносное или передвижное устройство для тушения очагов пожара за счет выпуска запасенного огнетушащего вещества. Огнетушитель обычно представляет собой цилиндрический баллон красного цвета с соплом .

        Появление первых огнетушителей (еще в 1734 год) позволяло иметь всегда под рукой необходимый запас огнетушащего вещества и при необходимости локализировать и ликвидировать пожар на ранних этапах его

        По видам применяемых огнетушащих веществ:

        -порошковые; 

        -воздушно-пенные;   

        - углекислотные

        Изобретателем пенного огнетушителя считается россиянин Александр Лоран (1904 год).

        \textbf{Огнетушитель порошковый ОП-1(з)} 

        Применяется для тушения возгорания твердых, жидких и газообразных веществ (класса А, В, С или В, С в Также возможно их применение для тушения электроустановок, находящихся под напряжением меньше 1000 В)

        \textbf{Огнетушитель углекислотный ОУ-5 (ОУ-8)}
        
        при возгораниях в офисных и складских помещениях, где эксплуатируется оборудование  под напряжением свыше 1000 В.

        \textbf{Воздушно-пенный ОВП-4 (з)
        Воздушно-эмульсионный}
        
        предназначен для тушения следующих классов пожаров:

        А, В, С. Е - электрооборудование под напряжением до 1000 В.

        не предназначен для тушения металлов (класс D) и прочих материалов, способных гореть без доступа воздуха.

        \textbf{Мобильные средства пожаротушения}
        
        Пожарные автомобили общего применения применяются для тушения пожаров и проведения аварийно-спасательных работ в жилых домах

        Основные пожарные автомобили общего применения – это пожарные автомобили, предназначенные для доставки личного состава к месту вызова, тушения пожаров и проведения спасательных работ с помощью вывозимых на них огнетушащих веществ и пожарного оборудования, а также для подачи к месту пожара огнетушащих веществ от других источников

        Высота автолестницы и подъемника в среднем составляет 30 метров. Выбор того или иного средства для подъёма на определенную высоту зависит от обстановки, местности и сложности подхода к объекту.

        При работе со штатным огнетушащим оборудованием, лестницу выдвигают на 2/3 общей длины, угол наклона стрелы составляет 55÷60°.

        \section{Защита населения и территорий в ЧС природного характера.}

        \subsection{Наводнения.}

        Наводнение - это временное затопление значительной части суши в результате действий сил природы, которое причиняет, как правило, большой материальный ущерб и приводит к гибели людей и животных.
        Если затопление не сопровождается ущербом, это называется РАЗЛИВОМ. 

        Причины возникновения:

        Интенсивные осадки и таяние снегов;

        Ледяные заторы на реках, разрушение плотин;

        Тайфуны, ветровые нагоны и цунами на морском побережье.

        Наводнения  периодически наблюдаются на большинстве рек нашей страны и занимают первое место среди других стихийных бедствий по повторяемости, площади распространения и ущербу.

        Наводнения на реках Дальнего Востока и Сибири: Амуре, Зее, Бурее, Уссури и Лене всегда приниамет характер национального бедствия.

        Ущерб, причиняемый наводнением, связан с целым
        рядом поражающих факторов, важнейшими из которых являются:

        - быстрый подъём воды и резкое увеличение скорости течения, приводящие к затоплению территории, гибели людей и скота, уничтожению имущества, сырья, продовольствия, посевов, огородов и т.п.

        - низкая температура воды, пребывание в которой людей может приводить к заболеваниям и гибели;

        - снижение  прочности и срока службы жилых и производственных зданий;

        - смыв плодородной почвы и заиливание посевов. 

        \begin{center}
            \textbf{ Критерии, характеризующие наводнение} 
        \end{center}

        \begin{itemize}
            \item  максимальный уровень воды, м
            \item  максимальный расход воды, м³/c
            \item скорость подъёма воды, см/ч
            \item  скорость течения воды, м/c
            \item  площадь затопления, км²
            \item  продолжительность, ч (сут., нед., мес.)
        \end{itemize}
        
        \begin{center}
            \textbf{ По размерам и наносимому им ущербу различают небольшие, большие, выдающиеся и катастрофические наводнения.} 
        \end{center}

        Небольшое наводнение наносит незначительный материальный ущерб и почти не нарушает нормального течения жизни людей. Повторяемость их  примерно 1 раз в 5-8 лет и характерны они для малых рек.

        Большое наводнение сопровождается значительным материальным ущербом, в том числе и причиняемым населению. Часть населения, материальных ценностей и скота эвакуируется. Повторяемость – примерно 1 раз в 10-25 лет.

        Выдающееся наводнение охватывает крупную речную систему,
        \\приносит большой материальный и моральный ущерб. Возникает необходимость массовой эвакуации населения. Повторяемость таких наводнений – примерно 1 раз в 50-100 лет.

        Катастрофическое наводнение распространяется на несколько крупных речных бассейнов. Оно надолго парализует хозяйственную деятельность  человека. Сопровождается человеческими жертвами. Повторяемость 1 раз в 100-200 лет и реже.

        Нагонные наводнения (нагоны):

        Ветровые нагоны воды в морских устьях рек и на ветреных участках побережья морей, крупных озёр, водохранилищ возможны в любое время года. Характеризуются отсутствием периодичности и значительным подъёмом уровня воды.

        Одним из самых опасных является наводнение, причина которого в прорыве плотины, дамбы или другого гидротехнического сооружения, либо в переливе воды через плотину из-за переполнения водохранилища. Затопление местности, расположенной ниже сооружения, осуществляется в этом случае внезапно, с приходом так называемой волны прорыва 

        \begin{center}
            \textbf{Причины возникновения.} 
        \end{center}

        Разрушение плотины или дамбы может происходить
        по естественным причинам или из-за деятельности человека.

        К природным (естественным) силам, способным вызвать прорыв гидротехнического объекта относятся: землетрясения, паводки, сильные и продолжительные ливни, ураганы, оползни.

        Естественная коррозия бетонных конструкций также способна привести к аварии, но сейчас чаще всего распространены грунтовые плотины.

        Различные неточности в проектировании, ошибки при сооружении объектов, дефекты материала или его низкое качество, взрывы, диверсии, военные действия вблизи гидродинамических сооружений относятся к причинам, которые связаны с человеческой деятельностью.

        При обнаружении хоть малейшего риска прорыва плотины производят действия по ее укреплению и предотвращению прорыва. Во время весенних паводков осуществляется регулярный сброс воды из объекта. 

        \begin{center}
            \textbf{Воздействие наводнения на население и окружающую среду.} 
        \end{center}
        Прямой ущерб:

        - гибель, переохлаждения и травмы людей;

        - повреждения и разрушения жилых и производственных зданий, дорог, линий электропередач и связи;

        - гибель скота и урожая;

        - уничтожение и порча сырья, топлива, продовольствия, кормов и удобрений;

        - затраты на временную эвакуацию населения, уничтожение плодородного слоя почвы.

        Косвенный ущерб:

        - затраты на приобретение и доставку в районы бедствия продуктов питания, кормов и необходимых материальных средств;

        - сокращение выработки продукции вследствие затопления предприятий;

        - ухудшения условий жизни населения;

        - невозможность рационального использования территорий в зоне затопления и другие.
    
        Наводнения в большинстве случаев доступны для прогнозирования, что позволяет предотвратить массовые жертвы среди населения и сократить ущерб.

        \subsection{Cпецифика мероприятий по защите населения и территорий в условиях наводнения.}

        Мероприятия по защите населения и территорий, проводимые заблаговременно в режиме повседневной деятельности:

        \begin{itemize}
            \item правовые:
            
            Руководство положениями основных документов в области защиты населения и территорий (21 декабря 1994 года N 68-ФЗ) применительно к наводнениям, а также рядом специальных документов, таких как ФЗ «О безопасности гидротехнических сооружений» и др.
            \item Организационные 
            
            Планирование защиты населения и территорий в условиях наводнения осуществляется в соответствии с общими положениями с учётом специфики наводнений.

            Особое внимание уделяется планированию
            эвакуации населения из зон затопления.
            \item Инженерно-технические 
            
            Регулирование паводкового стока (нагонных наводнений) с помощью гидротехнических сооружений (плотин, дамб), укрепления берегов рек, спрямление русел рек и подсыпка низменных участков территорий.
            \item Медико-профилактические
        \end{itemize}
        
        \subsection{Правила поведения при утоплении.}

        Последние пять лет в  России на воде погибло более 63 тысяч человек,
        свыше 14 тысяч из них - дети младше 15 лет.
        По состоянию на 28 мая 2022 года на территории Архангельской области было зарегистрировано 35 происшествий на воде, в которых 31 человек погиб и 7 человек пострадало
        В 2020 году на водных объектах Архангельской области произошло 101 происшествие, в которых погибли 103 человека (по данным Главного       управления МЧС России по Архангельской области).
         По данным Всемирной организации здравоохранения в мире на воде ежегодно гибнет около 450 тысяч человек. В странах с морским побережьем и теплым климатом утопление стоит на втором месте после дорожно-транспортных происшествий. Из общего числа утонувших 54% составляют    лица в возрасте 20-25 лет, большая часть которых умела плавать.
         Утопление – терминальное состояние или наступление смерти    вследствие аспирации (проникновения) жидкости в дыхательные пути, рефлекторной остановки сердца в холодной воде либо спазма голосовой щели, что в результате приводит к снижению или прекращению    газообмена в легких.
                                             Запомните!
         
        Находясь у воды, никогда не забывайте о собственной безопасности
        и будьте готовы оказать помощь попавшему в беду!
         
         Смерть от утопления обусловлена следующими причинами:
         
        Страх - один из ведущих факторов гибели людей при катастрофах на   воде. Возникающая у человека паника приводит к дискоординации движений.     В результате он либо захлёбывается водой, либо выбивается из сил и,             погружаясь в воду, делает непроизвольный вдох.
         
        Переохлаждение. Длительность безопасного пребывания в воде зависит от её температуры. Например, при 24 гр.С можно выжить, находясь в воде      до 8 часов, при 20 гр.С - 2,5 часа, при 15 гр.С - 1 час, при 10 гр.С - 35 минут. При температуре воды 4 - 6 гр.С уже через 10 - 20 минут появляются  нарушения двигательной способности и только 50% пострадавших выживают в      условиях такого режима.
         
        Другие причины утопления: неумение плавать; назо- или ларингокардиальный рефлекс при попадании воды в нос; травмы головы и шеи,  полученные при прыжке в воду; баротравма (при нырянии с аквалангом); переедание, алкогольная интоксикация; состояния, которые могут  сопровождаться потерей сознания (эпилепсия, нарушения ритма сердца, сахарный диабет и др.); скорость течения воды, наличие водоворотов и т.д.
         
        Первая помощь при утоплении
         Необходимо всегда помнить о собственной безопасности и в первую   очередь минимизировать непосредственную угрозу для себя. Поэтому, по возможности, старайтесь спасать тонущего человека, не заходя в воду.
        Если утопление происходит недалеко от берега, следует, поддерживая словесный контакт с жертвой, протянуть утопающему палку, бросить верёвку или любой плавучий предмет, который можно будет эффективно использовать.
         
        
        
        Этапы оказания помощи
         Выделяют два этапа оказания помощи при утоплении. Первый - это    действия спасателя непосредственно в воде, когда утопающий еще в сознании, предпринимает активные действия и в состоянии самостоятельно держаться на поверхности. В этом случае есть реальная возможность не допустить трагедии и отделаться лишь «легким испугом». Но именно этот вариант представляет наибольшую опасность для спасателя и требует от него прежде всего умения плавать, хорошей физической подготовки и владения специальными приемами подхода к тонущему человеку, а главное - умения освобождаться от «мертвых» захватов.
        При оказании первой помощи на воде необходимо помнить некоторые особенности поведения утопающего, а именно: судорожные, неосознанные, нескоординированные движения. Подплывать к утопающему надо сзади, чтобы он не мог обхватить вас руками. Просунув руки через подмышки или держа за волосы, надо повернуть его лицом вверх, и плыть к берегу.
                                             Запомните!
        Панический страх утопающего - смертельная опасность для спасателя!
         Искусственное дыхание во время буксировки утопающего в  бессознательном состоянии может оказаться намного полезней, чем последующее искусственное дыхание на берегу.
        Однако, выполнить на воде эти приёмы может только хорошо подготовленный и физически сильный спасатель.
        В том случае, когда из воды извлекается уже «бездыханное тело» -  пострадавший находится без сознания, а зачастую и без признаков жизни, -       у спасателя, как правило, нет проблем с собственной безопасностью, но значительно снижаются шансы на спасение. Исход будет зависеть от времени года, температуры и состава воды, особенностей организма, а главное - от вида утопления и верно выбранной тактики оказания помощи.   
         
        Утопления — это достаточно обширная группа угрожающих жизни     состояний, связанных с экс­тремальным воздействием водной среды, отлича­ющихся друг от друга как по механизму разви­тия, так и по характеру изменений, наступающих в организме пострадавшего. Эффективность пер­вой помощи во многом зависит от знания особенностей неотложных мероприятий при различных вариантах несчастных случаев на воде.
        Смертельная доза аспирированой воды для взрослого - 22 мл/кг; у 85\% пострадавших смерть наступает при аспирации 10 мл/кг. Аспирация 2 - 3 мл/кг приводит к потере сознания.
         Виды (типы) утоплений:
         1. Истинное («мокрое») утопление
        Составляет около 70-80\% всех случаев утопления. Истинное («мокрое») утопление характеризуется попаданием воды в трахеобронхиальное дерево,   когда после погружения в воду утопающий совершает непроизвольные  дыхательные движения.
        Привлечение плазмы крови в альвеолы способствует пенообразованию, пенистые выделения изо рта и носа носят обильный характер. Обращает на себя внимание резкий цианоз кожи.
         
        2. Асфиксическое («сухое») утопление
        Развивается в 10-15\% случаев утопления. Асфиксическое утопление    происходит без аспирации воды. Вода, попадая в гортань, вызывает   рефлекторный ларингоспазм, который приводит к асфиксии. Большое  количество воды заглатывается в желудок.
        В лёгких остается воздух, образуется мелкопузырная пена, которая   скапливается в уголках рта. Цианоз при этом типе утопления столь же выражен, как и при истинном («синие утопленники»).
         
        3. Синкопальное утопление («смерть в воде»)
        От слова «синкопе» - обморок. Также встречается в 10-15\% случаев. Смерть наступает в результате рефлекторного прекращения сердечной и дыхательной деятельности из-за перепада температур вследствие погружения в холодную воду («ледяной шок», «синдром погружения»), рефлекторной  реакции на попадание воды в дыхательные пути или полость среднего уха при повреждённой барабанной перепонке.
        Помимо обморока к утоплению также может привести потеря сознания, обусловленная приступом эпилепсии, инфарктом миокарда, аритмией и т.д. Все эти случаи можно классифицировать как «смерть в воде», когда человек      умирает от причин, не связанных напрямую с утоплением, но совпавших по времени с погружением в воду.
        Полость рта и носа свободна, пенистых выделений нет. В отличие от   первых двух типов, где наблюдается синюшность, обусловленная дыхательной недостаточностью, при синкопальном утоплении кожа бледная из-за  выраженного спазма периферических сосудов.
          
        Запомните!
        На успех можно надеяться только при правильном оказании
        помощи с учетом типа утопления.
        Признаки утопления
         
        Состояние извлеченных из воды пострадавших во многом определяется длительностью пребывания под водой и степенью охлаждения.
        В лёгких случаях сознание может быть сохранено, но больные   возбуждены. Отмечаются шумное дыхание с приступами кашля, рвота  проглоченной водой, дрожь.
        При длительном утоплении пострадавший может быть извлечен из воды без признаков дыхания и сердечной деятельности.
          
        В случае утопления действовать следует быстро.
        Любое промедление грозит обернуться страшной трагедией!
        Поэтому только своевременное оказание первой помощи
        при утоплении может спасти человеческую жизнь!
         
         Первая помощь при утоплениях
         
        Характер помощи пострадавшему зависит от периода утопления.
        В начальном периоде  первая помощь дол­жна быть направлена на устранение последствий психической травмы и переохлаждения.  Постра­давшие, спасенные в этом периоде утопления, со­храняют сознание и не нуждаются в мерах реанимации, но они должны находиться под контролем окружающих, так как возможны психические рас­стройства и неадекватные    реакции на обстановку. Со спасенного снимают мокрую одежду, насухо  обтирают, переодевают и укутывают в теплое оде­яло, дают горячее сладкое  питье. Синюшность кож­ных покровов, одышка и учащенный пульс обыч­но быстро проходят, но слабость, головная боль и кашель сохраняются несколько дней. Существует опасность рвоты проглоченной водой и желудоч­ным          содержимым, поэтому необходимо принимать меры, направленные на то, чтобы рвотные массы не попали в верхние дыхательные пути (если  пост­радавший находится в горизонтальном положении, его голова должна быть повернута на бок).
         В периоды агонии и клинической смерти при оказании помощи  нередко допускаются ошибки, связанные с попытками полностью освободить ле­гочную ткань от воды. Это приводит к значитель­ной потере времени, что снижает вероятность ус­пешной реанимации или полностью исключает  воз­можность восстановления жизненных функций организма.
         Первая помощь на берегу
         Как только пострадавший будет извлечён из воды, нужно проверить наличие дыхания. Профессионал также должен проверить пульс на сонной    артерии, но это может вызвать затруднения, особенно если утопление  произошло в холодной воде.
         В случае клинической смерти (отсутствие сознания, дыхания,  кровообращения, реакции зрачка на свет) - начать реанимационные     мероприятия. Реанимация при утоплении проводится по общим правилам, но есть некоторые отличительные особенности (см. ниже).
        Не нужно тратить время на удаление воды из лёгких
        - это бесполезно!
         
         У большинства утонувших аспирируется небольшое количество воды, которое (особенно при утоплении в пресной воде) быстро всасывается в кровь.
         Помимо того, что попытки удаления воды из дыхательных путей   задерживают проведение реанимации, они ещё могут быть опасными.  Например, брюшные толчки (приём Геймлиха) приводят к регургитации    желудочного содержимого с последующей его аспирацией.
        
        Первая помощь при утоплениях
           С помощью марлевой салфетки, намотанной на палец, очищат полость рта и ротоглотки от грязи (тины) и остатков желу­дочного содержимого.
          Сердечно-ле­гочная реанимация
         
        После этого незамедлитель­но приступают к проведению искусственной венти­ляции легких и закрытого массажа сердца в соот­ветствии со   стандартными правилами сердечно-ле­гочной реанимации.
        В ходе проведения реанимационных мероприятий обязательно согревание пострадавшего.
         
        Схема базовых реанимационных мероприятий
         Став свидетелем клинической смерти, либо обнаружив человека в бессознательном состоянии, необходимо выполнить определённую  последовательность действий:
        1. Подумать о собственной безопасности.
        2. Громким криком позвать на помощь.
        3. Оценить реакцию на внешние раздражители и попытку речевого     контакта: легко встряхнуть за плечи и громко окликнуть «Вы в порядке?» Не следует встряхивать голову и шею, если не исключена их травма.
        4. Обеспечить проходимость дыхательных путей. Для обеспечения  свободной проходимости дыхательных путей пациента следует положить на спину, без возвышения головы и подкладывания валика под лопатки. Открыть дыхательные пути при помощи следующих приёмов (другое название этих    манипуляций - тройной приём Сафара):
        - запрокидывание головы - одна рука размещается на лбу, и мягко  отклоняет голову назад; кончики пальцев другой руки размещаются под        подбородком или под шеей и мягко тянут вверх;
        - выдвижение вперёд и вверх нижней челюсти - четыре пальца   помещаются позади угла нижней челюсти и давление прикладывается вверх и вперёд; используя большие пальцы, приоткрывается рот небольшим смещением   подбородка.
        Каждый раз, запрокидывая голову пострадавшему, следует одновременно осмотреть его рот и, увидев инородное тело (например, обломки зубов или   выпавший зубной протез), удалить его. Приём очищения ротовой полости пальцами вслепую больше не применяется. Съёмные зубные протезы, которые держатся на месте, не удалять, т.к. они формируют контуры рта, облегчая    герметизацию при вентиляции.
        У пациента с подозрением на травму шейного отдела позвоночника      используется только выдвижение нижней челюсти (без запрокидывания  головы). Но, если не удаётся обеспечить свободную проходимость   дыхательных путей при помощи этого приёма, то следует выполнить   запрокидывание головы, не взирая травму, поскольку достижение адекватной вентиляции лёгких является приоритетным действием при реанимации  травмированных пациентов. При наличии достаточного количества спасателей один из них должен вручную обеспечить стабилизацию движения головы  пострадавшего по осевой линии, чтобы минимизировать наносимый вред.
        5. Проверить адекватность дыхания. Необходимо потратить не более        5 секунд на проверку наличия нормального дыхания у взрослого без сознания. Сохраняя дыхательные пути открытыми (см. пункт 4) применяют приём «Вижу, слышу, ощущаю»: ищут движения грудной клетки, слушают дыхательные    шумы изо рта пациента, пытаются ощутить воздух на своей щеке.
         6. Проверить пульс на сонной артерии. Необходимо потратить не более   5 секунд на определение пульса на сонных артериях. Если есть сомнения в наличии/отсутствии пульса, а у пациента отсутствуют другие признаки жизни (реакция на оклик, самостоятельное дыхание, кашель или движения),  то необходимо начать сердечно-легочную реанимацию.
        Если дыхание отсутствует (см. пункт 5), но есть пульс на сонной артерии, то необходимо начать искусственное дыхание с частотой 10 вдуваний воздуха  в минуту и повторно проверять пульс через каждые 10 вдуваний.
        Констатация остановки дыхания и кровообращения должна проводиться достаточно быстро. Вся диагностика клинической смерти (пункты 5 и 6) не должна занимать более 10 секунд. Задержка с распознаванием клинической смерти и промедление с началом реанимации неблагоприятно сказываются  на выживании и должны быть устранены.
        7. Приступить к выполнению непрямого массажа сердца. Непрямой    массаж сердца по современным представлениям играет первостепенную роль в оживлении, поэтому сердечно-легочная реанимация взрослых начинается с компрессий грудной клетки, а не с искусственного дыхания, как было раньше.При осуществлении непрямого массажа сердца следует выполнять    сильные и быстрые ритмичные толчки с глубиной надавливания в 4 - 5 см и  с частотой надавливаний на грудную клетку 100 в минуту. При этом надо  обеспечить выпрямление грудной клетки после каждого надавливания для наполнения сердца кровью, следя за тем, что продолжительность компрессии и декомпрессии грудной клетки была приблизительно одинаковой.
        Крайне важно как можно реже прерывать непрямой массаж сердца (паузы для вдувания воздуха или проверки пульса не должны превышать 10 секунд). Каждый раз, когда непрямой массаж останавливается, кровообращение также прекращается. Чем чаще прерывается непрямой массаж сердца, тем хуже  прогноз на выживание.
        Непрямой массаж сердца с указанными выше требованиями - это тяжёлая физическая работа, быстро вызывающая утомление, которое ведёт к снижению качества компрессий грудной клетки. Учитывая важность непрямого массажа сердца, его следует выполнять поочерёдно, если реанимацию оказывает 2 и   более медицинских работника. Каждые 2 минуты или каждые 5 циклов сердечно-легочной реанимации реаниматор, выполняющий непрямой массаж сердца, должен быть сменён. Смена спасателей должна занимать менее 5  секунд.
        8. Выполнить 2 вдувания воздуха методом «рот в рот» (метод «рот в нос» у взрослых не применяется) после 30 надавливаний на грудную клетку.
        Снова «открывают» дыхательные пути (см. пункт 4). Указательным и большим пальцами одной руки зажимают нос пациента, пальцами другой руки поддерживают его подбородок, делают обычный (неглубокий) вдох, герметично обхватывают своими губами рот пациента («поцелуй жизни») и осуществляют выдох. Поддерживая запрокинутую голову и выдвинутую  челюсть, убирают свои губы, чтобы воздух мог пассивно выйти из дыхательных путей пациента. Выполняют второй выдох и возвращаются к  непрямому массажу сердца.
        Вдувание воздуха должно длиться 1 секунду и сопровождаться видимой экскурсией грудной клетки. Выдох не должен быть слишком большим или  резким. Объём вдуваемого воздуха должен составлять 500 - 600 мл.
        Настоятельно рекомендуется применять барьерные приспособления, уменьшающие опасность передачи заболеваний в ходе искусственного дыхания «рот в рот». В первые минуты используют те защитные приспособления, которые находятся под рукой и позволяют избежать прямого контакта, например, носовой платок.
         9. Соотношение компрессий грудной клетки и вдуваний. В 2005 году установлено новое единое соотношение количества компрессий грудной клетки и вдуваний независимо от количества реаниматоров как 30:2
        
        Непрямой массаж сердца и искусственная вентиляция лёгких в соотношении 30:2 продолжается до тех пор, пока не прибудет бригада скорой помощи   или пациент не начнёт проявлять признаки жизни.
         1. Проверить ответную реакцию:
         Громко окликнуть. 
         Легко встряхнуть за плечи.
         2. Если ответная реакция не получена:
         Открыть дыхательные пути.
         Проверить адекватность дыхания.
         3. Если человек без сознания, но дыхание нормальное:
         Придать устойчивое боковое положение. 
         Позвонить в Службу спасения - 01.
         Регулярно повторно проверять дыхание.
         4. Если нормальное дыхание отсутствует:
          Позвонить в Службу спасения - 01.
          начать сердечно-легочную реанимацию.
         5. Непрямой массаж сердца:
         Поместить руки в центр грудной клетки   (нижняя 1/3 грудины).
         Сделать 30 сильных надавливаний на грудину.
         Грудина прогибается на 3-4 см.
         Частота компрессий - 100 в 1 минуту.
                                                                                    
        6. Искусственное дыхание:
         Обхватить губами рот и сделать 2 выдоха.            
         Вдувание воздуха должно вызывать подъём грудной клетки (1 вдох-1 сек.).
         Продолжить реанимацию в соотношении 30 к 2.
          Если человек начал  нормально дышать - прекратить реанимацию.
          Если  при  наличии дыхания человек не пришел в сознание – придать устойчивое боковое  положение.
        Критериями правильно оказанной помощи яв­ляются:
        - появление самостоятельного дыхания;
        - появление сердцебиения;
        - восстановление реакции зрачков на свет.
        Если человек пришел в сознание, дыхание и пульс удовлетворительны, то пострадавшего нужно уложить на сухую жесткую поверхность. Голова  пострадавшего должна быть низко опущена. Пострадавшего следует избавить от   стесняющей одежды, растереть руками или полотенцем. Дать  пострадавшему горячее питье, укутать теплым одеялом. Вызвать «скорую»  и обязательно  отправить пострадавшего на госпитализацию, поскольку даже после  восстановления жизненных функций остается риск развития вторичного утопления и отека легких.
         Особенности проведения реанимации утонувших:
         1. Сердечно-легочную реанимацию при утоплении надо проводить даже в том случае, если человек находился под водой в течение 10-20 минут  (особенно если речь идёт об утоплении ребёнка в холодной воде). Поскольку описаны случаи оживления с полным неврологическим восстановлением при нахождении под водой более 60 минут.
        2. Если во время сердечно-легочной реанимации произошёл заброс  содержимого желудка в ротоглотку, следует повернуть реанимируемого на бок (при возможной травме шейного отдела позвоночника – следить за тем, чтобы взаиморасположение головы, шеи и туловища не изменились), очистить рот, а затем повернуть обратно на спину и продолжить реанимационные         мероприятия.
        3. При подозрении на повреждение шейного отдела позвоночника  рекомендуется попытаться обеспечить свободную проходимость дыхательных путей, используя приём «выдвижения вперёд нижней челюсти» без запрокидывания головы пострадавшего. Но, если с помощью этого приёма не удаётся обеспечить свободную проходимость дыхательных путей, то разрешено применять запрокидывание головы даже у пациентов с подозрением на травму шейного отдела позвоночника, поскольку обеспечение свободной  проходимости дыхательных путей остаётся приоритетным действием при реанимации травмированных пациентов в бессознательном состоянии.
        4. Одной из наиболее частых ошибок при проведении сердечно-легочной реанимации является преждевременное прекращение искусственного дыхания. Прекращать его можно только после полного восстановления сознания и исчезновения признаков дыхательной недостаточности. Искусственное  дыхание необходимо продолжать в том случае, если у пострадавшего имеются нарушения ритма дыхания, учащение дыхания (более 40 в минуту) или резкий цианоз.
         Прогноз:
        Сиюминутный успех неотложной помощи не страхует от возможных поздних осложнений.
        Выжившие после утопления имеют высокий риск развития острого       респираторного дистресс-синдрома в ближайшие 72 часа. Бурный отёк лёгких (основная причина смерти) довольно часто наступает в первые 8 - 24 часа. Ранее этот симптомокомплекс называли «вторичным утоплением».
        Таким образом, о спасении от утопления можно говорить, еслипострадавший прожил не менее 24 часов после извлечения из воды.
         Признаки плохого прогноза:
        Длительное пребывание под водой (более 25 минут)
        Остановка дыхания
        Проводилась сердечно-легочная реанимация (даже непродолжительная)
        Наличие цианоза
        Кома на момент госпитализации
        У большинства пострадавших, как правило, на 1-3 сутки после утопления развиваются ателектазы и пневмония (чаще аспирационная). Но профилактическое применение антибиотиков не показано, если только  утопление не произошло в инфицированной воде (например, в непроточных водоёмах)
         Профилактика утопления:
        Не следует употреблять перед купанием пищу и алкогольные   напитки! На пляже и возле водоёмов нельзя оставлять детей без надзора и надо, как можно раньше, научить их плавать. Нельзя оставлять в ванне без   присмотра   детей, инвалидов и стариков.
        Плавать следует только в зоне спасательной станции. Не умеющие плавать должны одевать спасательные жилеты или использовать надувной круг.
         
        \section{Защита населения и территорий в чрезвычайных ситуациях, обусловленных экстремизмом}

        \subsection{Общие сведения об экстремизме.}

        ЭКСТРЕМИЗМ (от фр. extremisme, от лат. extremus — крайний) — приверженность к крайним взглядам и, в особенности, мерам.

        Федеральный закон № 114 от 25.07.02г. ( ред. 28 декабря 2022 г. ) «О противодействии экстремистской деятельности» дает нам достаточно широкое понятие экстремизма – это:

        - насильственное изменение основ конституционного строя и нарушение целостности Российской Федерации;

        - публичное оправдание терроризма и иная террористическая деятельность;

        - возбуждение социальной, расовой, национальной или религиозной розни;

        пропаганда исключительности, превосходства либо неполноценности человека по признаку его социальной, расовой, национальной, религиозной или языковой принадлежности или отношения к религии;

        - нарушение прав, свобод и законных интересов человека и гражданина в зависимости от его социальной, расовой, религиозной или языковой принадлежности или отношения к религии;

        - воспрепятствование осуществлению гражданами их избирательных прав и права на участие в референдуме или нарушение тайны голосования, соединенные с насилием либо угрозой его применения;

       -  воспрепятствование законной деятельности государственных органов, органов местного самоуправления, избирательных комиссий, общественных и религиозных объединений или иных организаций, соединенное с насилием или угрозой его применения;

        - совершение преступления по мотивам политической, идеологической, расовой, национальной или религиозной ненависти или вражды либо по мотивам ненависти или вражды в отношении какой-либо социальной группы;

        - организация и подготовка указанных деяний, а также подстрекательство к их осуществлению;

        - финансирование указанных деяний   либо иное содействие     в их       организации, подготовке и осуществлении,                  в том числе путем предоставления учебной,       полиграфической и материально-технической базы,      телефонной и иных видов связи или оказания информационных услуг.

        \subsection{Терроризм. Классификация терроризма.}

        Терроризм — политика, основанная на систематическом применении террора. Несмотря на юридическую силу термина «терроризм», его определение в плоть до настоящего времени остается неоднозначным.  Но специалисты сходятся во мнении, что лучшим определением терроризма является достижение политических, идеологических, экономических и религиозных целей насильственным путем. Синонимами слова «террор» (лат. terror — страх, ужас) являются слова «насилие», «запугивание», «устрашение». 

        Терроризм как крайняя форма проявления экстремизма и радикализма он разнообразен, многолик, имеет различную природу, разные источники, цели, разные уровни и масштабы, направленность и характер исполнения. 

        Принципиальная общность всех видов терроризма: 

        – в насильственном насаждении мировоззрения, идеологии, морали, политики, своего образа жизни, 

        - в использовании убийств мирных жителей,       угрозы убийств или других форм насилия в качестве главного средства достижения целей.

        Рассмотрим факторы влияющие на рост терроризма в нашей стране начиная с 1991года.

        К числу внутренних факторов роста терроризма, на наш взгляд, относятся:

        -ослабление или отсутствие ряда административно-контрольных правовых режимов;

        -сплоченность и иерархичность преступной среды;

        -утрата многими людьми идеологических и духовных жизненных ориентиров;

        -обостренное чувство социальной неустроенности, незащищенности у значительных контингентов граждан;

        -настроения отчаяния и рост социальной агрессивности, общественная\\ фрустрация, падение авторитета власти и закона, веры в способность и возможность позитивных изменений;

        -слабая работа правоохранительных и социальных государственных и общественных органов по защите прав граждан; 
        
        - широкая пропаганда (кино, телевидение, пресса, литература) культа жестокости и силы.

        К числу внешних факторов, влияющих на распространение терроризма, следует отнести: 

        -рост числа террористических проявлений в ближнем и дальнем зарубежье; социально-политическую и экономическую нестабильность в сопредельных государствах как бывшего СССР, так и Европы и Восточной Азии; 

        -наличие вооруженных конфликтов в отдельных из них, а также территориальных претензий друг к другу; стратегические установки некоторых иностранных спецслужб и зарубежных (международных) террористических организаций; 

        -отсутствие надежного контроля за въездом-выездом из России и\\ сохраняющуюся "прозрачность" ее границ; наличие значительного "черного\\ рынка" оружия в стране и
        некоторых сопредельных государствах.

        террористический акт - совершение взрыва, поджога или иных действий, связанных с устрашением населения и создающих опасность гибели человека, причинения значительного имущественного ущерба либо наступления экологической катастрофы или иных особо тяжких последствий, в целях противоправного воздействия на принятие решения органами государственной власти, органами местного самоуправления или международными организациями, а также угроза совершения указанных действий в тех же целях

        \subsection*{Классификация терроризма.}

        \textbf{Политический} -- акции, осуществляемые подпольными группами против государственных органов и высших должностных лиц.
        
        \textbf{Этнический.} К террору как к способу борьбы за государсвенную независимость или предоставление широкой автономии иногда прибегают представители этнических меньшинств.
        
        Акции устрашения осуществляют национально-освободительные движения, ведущие войну с колонизаторами и странами-агрессорами.

        \textbf{Религиозный} -- напрямую связан с этническим терроризмом и очень трудно понять, где заканчивается один и начинается другой. Но если этнические террористы ведут войну за свою историческую территорию. 

        \textbf{Индивидуальный.} Одиночки, идущие на совершение террористического акта, руководствуются самыми разными мотивами -- политического, этнического, религиозного и иного характера.
        
        Нередко истинными заказчиками преступления являются тайные организации, которым выгодно, чтобы ответственность за акт террора легла на одного человека.

        \textbf{Криминальный} -- имеет часто экономические причины. Бандиты стремятся запугать чиновников, коммерсантов или даже целые организации, чтобы заставитьих принять свои требования -- выплачивать "криминальный налог", передать бизнес под контроль той или иной преступной группировке.

        Силовые акции устраиваются также против тех представителей власти и закона, которые мешают организационной преступности. Бандиты идут на убийства, проводят диверсии на предприятиях, устраивают взрывы на многолюдных рвнках, в ресторанах, кафе и торговых центрах. С целью выкупа захватывают заложников.

        Также классифицируются:
        \begin{itemize}
            \item По масштабу (внутренний, международный).
            \item По количеству применяемых сил и средств (индивидуальный, групповой, массовый).
            \item По целям и задачам (меркантильный, аппокалиптический)
            \item По видам применяемых средств (обычный, ядерный, химический, биологический, кибернетический, информационный, экономический).
        \end{itemize}
        
        \subsection{Основные принципы противодействия терроризму.}

        Противодействие терроризму в Российской Федерации основывается на следующих основных принципах:

        1) обеспечение и защита основных прав и свобод человека и гражданина;

        2) законность

        3) приоритет защиты прав и законных интересов лиц, подвергающихся террористической опасности;

        4) неотвратимость наказания за осуществление террористической деятельности;

        5) системность и комплексное использование политических, информационно-пропагандистских, социально-экономических, правовых, специальных и иных мер противодействия терроризму;

        6) сотрудничество государства с общественными и религиозными объединениями, международными и иными организациями, гражданами в противодействии терроризму;

        7) Приоритет мер предупреждения терроризма.

        8) сочетание гласных и негласных методов противодействия терроризму;

        9) конфиденциальность сведений о спец. средствах, технических приемах, тактике осуществления мероприятий по борьбе с терроризмом, а также о составе их участников;

        10) недопустимость политических уступок террористам;

        11) минимизация и (или) ликвидация последствий проявлений терроризма;

        12) соразмерность мер противодействия терроризму степени террористической опасности.

        \begin{center}
            \textbf{Порядок установления уровней террористической опасности } 
        \end{center}

        Повышенный "синий" уровень вводится при получении информации о возможной подготовке теракта. Спецслужбы в это время проводят специальные мероприятия, но никаких дополнительных ограничений для простых граждан не вводится.

        ВЫСОКИЙ "Желтый" уровень объявляют тогда, когда оперативная информация о теракте подтвердилась. Об этом предупреждают население, рекомендуя воздерживаться от посещения людных мест. Кроме того, люди должны быть готовы к усилению паспортного режима и проверкам транспорта.

        КРИТИЧЕСКИЙ- "красный" уровень вводится, когда теракт совершен и\\ ожидаются новые. В этом случае отменяются все массовые мероприятия, а при необходимости временно закрываются школы и детсады, приостанавливается работа общественного транспорта. Правоохранительные органы и спецслужбы получают максимальные полномочия по борьбе с терроризмом

        \subsection{Оповещение и информация населения о террористических актах}

        Осуществляется по существующей системе оповещения о ЧС и по средствам массовой информации. При наличии достоверной информации о возможных террористических актах население должно быть информировано об этом в \\кратчайшие сроки и с соответствующими инструкциями о правилах поведения в данной обстановке.

        \section{Защита населения и территорий при авариях на радиационно (ядерно) опасных объектах с выбросом радиоактивных веществ в окружающую среду}




        \section{Литература}
        \begin{enumerate}
            \item Каракеян В.И. "Безопасность жизенедеятельности: учебник и практикум для студентов."
            \item 21.12.1994 г. N 68-ФЗ "О защите населения и территорий от чрезвычайных ситуаций техногенного характера" с изм. 2023 г.
            \item Постановление Правительства РФ от 26.11.2007 г. N 804 "Положение о гражданской обороне в Российкой Федерации" в ред. 2023 г.
            \item Постановление Правительства РФ от 02.11.2000 г. N 841 "Об утвердении Положения о подготовке населения в области гражданской обороны" в ред. от 2023 г.
            \item Приказ МЧС России от 18.11.2015 г. N 601 "Положение об организации и введении гражданской обороны в муниципальных образованиях и организациях."
            \item Нормативно-инструкционные и справочные материалы с использованием информационно-поисковых систем, компьютерной сети "Интернет".
            \item Федеральный закон от 21 декабря 1994 г. N 69-ФЗ "О пожарной безопасности" (с изменениями и дополнениями от 2021гг.)
        \end{enumerate}
\end{document}