\documentclass[a4paper, 12pt]{article}

\usepackage{cmap}
\usepackage[T2A]{fontenc}
\usepackage[english, russian]{babel}
\usepackage[utf8]{inputenc}
\usepackage[left=2cm,right=1.5cm,top=2cm,bottom=2cm]{geometry}
% \usepackage{mathtext}
\usepackage{amsmath}
\usepackage{amssymb}
\usepackage{etoolbox}
\usepackage{amsthm}
\usepackage{booktabs}
% \usepackage{nicematrix}
\usepackage{graphicx}
% \usepackage{tikz}
% \usepackage{parskip}

%Реализация aug, overbrace и underbrace без nice matrix
\newcommand\aug{\fboxsep=-\fboxrule\!\!\!\fbox{\strut}\!\!\!}
\newcommand\undermat[2]{\makebox[0pt][l]{$\smash{\underbrace
{\phantom{\begin{matrix}#2\end{matrix}}}_{\text{$#1$}}}$}#2}
\newcommand\overmat[2]{\makebox[0pt][l]{$\smash{\overbrace
{\phantom{\begin{matrix}#2\end{matrix}}}^{\text{$#1$}}}$}#2}
\newcommand\tab[1][.5cm]{\hspace*{#1}}
\newcommand\Underset[2]{\underset{\textstyle #1}{#2}}
\newcommand\Overset[2]{\overset{\textstyle #1}{#2}}


\theoremstyle{definition}
\newtheorem*{definition}{Определение}
\newtheorem*{theorem}{Теорема}
\newtheorem*{consequense}{Следствие}
\newtheorem*{lemma}{Лемма}
\newtheorem*{subtheorem}{Утверждение}
\newtheorem*{remark}{Замечание}

\usepackage[russian]{babel}
\addto\captionsenglish{% Replace "english" with the language you use
  \renewcommand{\contentsname}%
    {Содержание}%
}

\usepackage{titlesec}
\titleformat{\section}{\LARGE \bfseries}{\thesection}{1em}{}
\titleformat{\subsection}{\Large\bfseries}{\thesubsection}{1em}{}
\titleformat{\subsubsection}{\large\bfseries}{\thesubsubsection}{1em}{}

\usepackage{hyperref}
\usepackage{xcolor}
% Цвета для гиперссылок
\definecolor{linkcolor}{HTML}{225ae2} % цвет ссылок
\definecolor{urlcolor}{HTML}{225ae2} % цвет гиперссылок
\hypersetup{
    pdfstartview=FitH, 
    linkcolor=linkcolor,
    urlcolor=urlcolor,
    colorlinks=true
}

% \title{\textbf{Безопасность жизенедеятельности}}
% \author{Ким Никита, 111 группа}

\begin{document}
    \fontsize{14pt}{20pt}\selectfont
    % \maketitle
    % \newpage
    % \tableofcontents
    % \fontsize{14pt}{20pt}\selectfont
    \begin{center}
        \begin{Large}
            \textbf{Проверочная работа 9}\\
            \textbf{Ким Никита 111 группа.} 
        \end{Large}
    \end{center}
    1. Ваша оценка складывающейся обстановки и
    значимость реализации задач ГО (перечислить) в
    настоящее время.

    \textit{На данный момент реализация задач гражданской обороны имеет огромное значение, ведь за последнее время произошло множество событий в связи с котрыми актуальны мероприятия по подготовке к защите и по защите населения. Наиболее важна подготовка населения в области гражданской обороны. Немаловажным является оповещение населения об опасностях, возникающих при военных конфликтах или вследствие этих конфликтов, а также при чрезвычайных ситуациях природного и техногенного характера; эвакуация населения, материальных и культурных ценностей в безопасные районы; проведение аварийно-спасательных и других неотложных работ в случае возникновения опасностей для населения при военных конфликтах или вследствие этих конфликтов, а также при чрезвычайных ситуациях природного и техногенного характера; Первоочередное жизнеобеспечение населения, пострадавшего при военных конфликтах или вследствие этих конфликтов, а также при чрезвычайных ситуациях природного и техногенного характера.}

    2. За год подготовки у Вас сложилась
    предварительная оценка реализации основных
    задач в области ГО на факультете (перечислите).
    На что необходимо, в плане подготовке по ГО,
    сосредоточить внимание руководству факультета
    (кафедре БЖД) и каждому сотруднику и студенту
    факультета?

    \textit{Очевидно, что лучше всего реализуется обучение по предмету БЖД, чтение памяток, листовок, пособий. Несложно заметить, что учения и тренировки по ГО не проводятся, что не странно, ибо мероприятия подобного характера наиболее часто реализуются в военных или иных специальных учреждениях. На что именно стоит сосредоточить внимание руководству факультета и сотрудникам в плане подготовки ГО сказать трудно не имея полной картины. Целью же каждого студента является изучение дисциплины БЖД.}

    3. Ваше мнение по совершенствованию УМБ по дисциплине БЖД (конкретные мероприятия).

    \textit{Вопросов или предложений по поводу учбно-матриальной бызы по курсу подготовки БДЖ нет.}

    4. По курсу подготовки БЖД определены девять
    тем. Какие темы вы считаете наиболее
    актуальными в наши дни? Есть ли необходимость
    в проведении практических занятий (пример)?

    \textit{Наиболее актуальной я считаю тему терроризма, ибо в достаточно больших населенных пунктах данное явление возкает часто. Также немаловажной темой является первая помощь. Лично я за последний год около 6 раз оказывал первую помощь. Именно по этой теме, возможно, стоит проводит практические занятия.}

    5. Ваша оценка пожарной безопасности факультета.
    Ваши предложения по пожарной безопасности
    (набор практических мер и правил, направленных
    на предотвращение возникновения случайного
    или преднамеренного пожара на факультете,
    ограничение его распространения).

    \textit{Оценивать пожарную безопасноть довольно сложно, ибо непосредственной какой-либо связи не имел. Из предложений: за все время обучения на мехмате я ни разу не заметил пожарного плана этажей ГЗ. Даже если они есть, то за все время ни я, ни мои знакомые их не видели. Это означает, что в этом случае их расположение не удачно. В случае если их нет, то нужно разместить, ибо за несоклько месяцев можно выучить расположение лестниц, помещений и т.д. Но первокурсники, что только прибыли на факультет, имеют возможность заблуждать. В худшем случае это может случиться во время оповещения о возгорании.}

\end{document}