\documentclass[a4paper, 12pt]{article}

\usepackage{cmap}
\usepackage[T2A]{fontenc}
\usepackage[english, russian]{babel}
\usepackage[utf8]{inputenc}
\usepackage[left=2cm,right=1.5cm,top=2cm,bottom=2cm]{geometry}
% \usepackage{mathtext}
\usepackage{amsmath}
\usepackage{amssymb}
\usepackage{etoolbox}
\usepackage{amsthm}
% \usepackage{nicematrix}
% \usepackage{graphicx}
% \usepackage{tikz}
% \usepackage{parskip}

% \graphicspath{/Images/}

%Реализация aug, overbrace и underbrace без nice matrix
\newcommand\aug{\fboxsep=-\fboxrule\!\!\!\fbox{\strut}\!\!\!}
\newcommand\undermat[2]{\makebox[0pt][l]{$\smash{\underbrace
{\phantom{\begin{matrix}#2\end{matrix}}}_{\text{$#1$}}}$}#2}
\newcommand\overmat[2]{\makebox[0pt][l]{$\smash{\overbrace
{\phantom{\begin{matrix}#2\end{matrix}}}^{\text{$#1$}}}$}#2}
\newcommand\tab[1][.5cm]{\hspace*{#1}}
\newcommand\Underset[2]{\underset{\textstyle #1}{#2}}
\newcommand\Overset[2]{\overset{\textstyle #1}{#2}}


\theoremstyle{definition}
\newtheorem*{definition}{Определение}
\newtheorem*{theorem}{Теорема}
\newtheorem*{consequense}{Следствие}
\newtheorem*{lemma}{Лемма}
\newtheorem*{subtheorem}{Утверждение}
\newtheorem*{remark}{Замечание}


\begin{document}
    \fontsize{14pt}{20pt}\selectfont
    \begin{center}
        \begin{Large}
            \textbf{Проверочная работа 7.}\\
            \textbf{Ким Никита 111 группа.} 
        \end{Large}
    \end{center}
    \textbf{1. Перечень состояний, при которых оказывается первая медицинская помощь:}
    \begin{itemize}
        \item Отсутствие сознания
        \item Остановка дыхания и кровообращения
        \item Наружные кровотечения 
        \item Инородные тела верхних дыхательныъ путей
        \item Травмы различных областей тела
        \item Ожоги, эффекты воздействия высоких температур,теплового излучения
        \item Отморожение и другие воздействия низких температур
        \item Отравления
    \end{itemize}
    \textbf{Перечень мероприятий по оказанию первой помощи:}
    \begin{itemize}
        \item Убедись, что ни тебе, ни пострадавшему ничто не угрожает. Используй медицинские перчатки для защиты от биологических жидкостей пострадавшего. Вынеси (выведи) пострадавшего в безопасную зону.
        \item Проверь признаки сознания у пострадавшего. При его наличии — перейди к пункту №7 и далее.
        \item При отсутствии сознания обеспечь проходимость верхних дыхательных путей и проверь признаки дыхания. При его наличии переходи к пункту №6 и далее.
        \item При отсутствии дыхания вызови (самостоятельно или с помощью окружающих) скорую медицинскую помощь (со стационарного телефона — 03,  с мобильного телефона — 112).
        \item Восстанови дыхание и сердечную деятельность путем надавливаний на грудную клетку и проведения искусственного дыхания  (30 надавливаний на 2 вдоха).
        \item В случае появления признаков жизни у пострадавшего (или в случае, если эти признаки имелись у него изначально) выполни поддержание  проходимости дыхательных путей (устойчивое боковое положение).
        \item Выполни обзорный осмотр пострадавшего. Останови наружное кровотечение при его наличии.
        \item Выполни подробный осмотр пострадавшего на наличие травм и неотложных состояний, окажи  первую помощь (например, наложи герметизирующую повязку на грудную клетку при проникающем ранении).
        \item Придай пострадавшему оптимальное положение тела, определяющееся его состоянием и характером имеющихся у него травм.
        \item До прибытия скорой медицинской помощи или других служб контролируй состояние пострадавшего, оказывай ему психологическую поддержку.
        \item  По прибытии бригады скорой медицинской помощи передай ей пострадавшего, ответь на вопросы и окажи возможное содействие.
    \end{itemize} 
    \textbf{2. Как правильно оказать ПП при обработке раны:}\\
    \textit{2.1. Остановка кровотечения:}
    \begin{itemize}
        \item Прямое давление на рану.
        \item Наложение давящей повязки. 
        \item Пальцевое прижатие артерии.
        \item Максимальное сгибание конечности в суставе.
        \item Наложение кровоостанавливающего жгута (табельного или импровизированного).
    \end{itemize}
    \textit{2.2. Обработка раны:}
    \begin{itemize}
        \item Осмотрите рану, определите, каков её характер и насколько сильное повреждение 
        \item Постарайтесь остановить кровотечение. 
        \item Промойте рану перекисью водорода (3\%), раствором хлоргексидина или фурацилина (0,5\%) или раствором марганцовки розового цвета (его нужно процедить через марлю). Осушите рану салфеткой. 
        \item Обработайте кожу вокруг раны антисептиком и наложите стерильную повязку. Впоследствии не забывайте делать перевязки. 
        \item Примите решение о необходимости обращения к доктору. При необходимости выпейте болеутоляющее средство. 
    \end{itemize} 
    \newpage
    \textit{2.3. Дезинфекция:} Последние три предыдущих пункта.\\
    \textit{Давящая повязка:}\\
    Для более продолжительной остановки кровотечения можно использовать давящую повязку. При ее наложении следует соблюдать общие принципы наложения бинтовых повязок: на рану желательно положить стерильные салфетки из аптечки, бинт должен раскатываться по ходу движения, по окончании наложения повязку следует закрепить, завязав свободный конец бинта вокруг конечности. Поскольку основная задача повязки – остановить кровотечение, она должна накладываться с усилием (давлением). Если повязка начинает пропитываться кровью, то поверх нее накладывают еще несколько стерильных салфеток и туго прибинтовывают.\\
    \underline{Порядок обработки раны:}
    \begin{enumerate}
        \item Дома: В домашних условиях есть возможность правильно обработать рану. В таком случае нужно выполнить пункты пречисленные выше.
        \item В лесу: Не в домашних условиях обработка раны затруднительна. Первым делом нужно платком, одеждой или другой тканью туго перевязать рану. В случае наличия аптечки, предвательно обработать рану, следуя пунткам выше. После нужно доставить постарадвшего до ближайшего места, где есть возможность правильно обработать рану. По мере надобности, вызвать скорую помощь.
    \end{enumerate}
    \textbf{3. Последовательность оказания ПП при ДТП.}\\
    Первым делом, вытащить постарадвшего из машины и соотвественно проезжей части. После чего аккуратно уложить на спину. "А" при помощи воды промывает рану и перевязывает ее. "Б" вызывает скорую помощь. Среднее прибытие скорой в Москве составляет 25 минут, за это время надо успеть остановить или замедлить кровотечение и, желательно, привести пострадавшего в чувство. Ни в коем случае не переворачивать раной к земле, это лишь ускорит кровотечение.
     
\end{document}